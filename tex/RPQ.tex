\setchapterpreamble[u]{\margintoc}
\chapter{Поиск путей с регулярными ограничениями}
\tikzsetfigurename{RPQ_}

\section{Достижимость между всеми парами вершин}

Через тензорное произведение.

Классическое построение пересечения автоматов строит их тензорное произведение.

Так как мы хотим отвечать ещё и на вопрос о достижимости, что нам надо ещё и транзитивное замыкание посчитать.

\section{Достижимость с несколькими источниками}

Достижимость от нескольких стартовых вершин через обход в ширину, основанный на линейной алгебре~\sidecite{9286186}.

Идея алгоритма основана на одновременном обходе в ширину графа и конечного автомата, построенного по грамматике.

В классической версии обхода в ширину, основанного на линейной алгебре, используется вектор, куда записывается фронт обхода графа.
Так, один раз перемножая этот вектор на матрицу смежности графа, можно совершать один шаг в обходе графа.
Покажем на примере, как данный метод может быть использован, когда мы накладываем дополнительные ограничения в виде регулярного языка на путь в графе.

Для этого, во-первых, предъявим булевые представления для матриц смежности графа и автомата для регулярного языка.
Затем, введем специальную блочно-диагональную матрицу для синхронизации обхода в ширину по двум матрицам смежности.
Далее, попробуем наивно реализовать обход в ширину, и посмотрим, почему наивная реализация может выдавать некорректный результат.
После этого перейдем к реализации обхода в ширину более продвинутым методом, который решает проблему наивного подхода.

\subsection{Пример работы алгоритма}
\marginnote{TODO: Подумать над разбиением на разделы и подразделы}
Возьмём следующий граф.
\begin{center}
    \label{input_rpq}
    \begin{tikzpicture}[node distance=2cm,shorten >=1pt,on grid,auto]
        \node[state] (q_0)   {$1$};
        \node[state] (q_1) [above=of q_0] {$2$};
        \node[state] (q_2) [right=of $(q_0)!0.5!(q_1)$] {$0$};
        \node[state] (q_3) [right=of q_2] {$3$};
        \path[->]
        (q_0) edge  node {b} (q_1)
        (q_1) edge  node[pos=0.3] {a} (q_2)
        (q_2) edge  node[pos=0.7] {a} (q_0)
        (q_2) edge[bend left]  node[above] {b} (q_3)
        (q_3) edge[bend left]  node {b} (q_2);
    \end{tikzpicture}
\end{center}

Его матрица смежности имеет следующий вид.
\marginnote{TODO: Точка внизу или снизу?}
\[
    G_1 =
    \begin{pmatrix}
        .     & \{a\} & .     & \{b\} \\
        .     & .     & \{b\} & .     \\
        \{a\} & .     & .     & .     \\
        \{b\} & .     & .     & .
    \end{pmatrix}
    \qquad
    G_1 =
    \begin{pmatrix}
        \cdot & \{a\} & \cdot & \{b\} \\
        \cdot & \cdot & \{b\} & \cdot \\
        \{a\} & \cdot & \cdot & \cdot \\
        \{b\} & \cdot & \cdot & \cdot
    \end{pmatrix}
\]

Её булева декомпозиция по каждому символу выглядит следующим образом.
\[
    G_{0\_a} = \begin{pmatrix}
        0 & 1 & 0 & 0 \\
        0 & 0 & 0 & 0 \\
        1 & 0 & 0 & 0 \\
        0 & 0 & 0 & 0 \\
    \end{pmatrix} \qquad
    G_{0\_b} = \begin{pmatrix}
        0 & 0 & 0 & 1 \\
        0 & 0 & 1 & 0 \\
        0 & 0 & 0 & 0 \\
        1 & 0 & 0 & 0 \\
    \end{pmatrix}
\]

Зададим ограничения с помощью регулярного выражения $b^*ab$, которое представляется автоматом из трех последовательных состояний.
\begin{center}
    \begin{tikzpicture}[shorten >=1pt,on grid,auto]
        \node[state, initial]   (q_0) at (0,0)  {$0$};
        \node[state]             (q_1) at (2,0)  {$1$};
        \node[state, accepting] (q_2) at (4,0)  {$2$};
        \path[->]
        (q_0) edge  node {$a$} (q_1)
        (q_1) edge  node {$b$} (q_2);
        \draw (q_0) edge[loop above]  node {$b$} (q_0);
    \end{tikzpicture}
\end{center}

Автомат может быть задан матрицей смежности (с дополнительной информацией о стартовых и финальных состояниях).

Для регулярного выражения $b^*ab$ матрица смежности выглядит следующим образом (при этом нужно запомнить, что состояние $0$ является начальным, $2$~--- конечным).
\[
    G_2 =
    \begin{pmatrix}
        \{b\} & \{a\} & .     \\
        .     & .     & \{b\} \\
        .     & .     & .
    \end{pmatrix}
\]

Нам будет необходима булева декомпозиция этой матрицы, и она выглядит следующим образом.
\[
    R_{0\_a} = \begin{pmatrix}
        0 & 1 & 0 \\
        0 & 0 & 0 \\
        0 & 0 & 0
    \end{pmatrix} \qquad
    R_{0\_b} = \begin{pmatrix}
        1 & 0 & 0 \\
        0 & 0 & 1 \\
        0 & 0 & 0
    \end{pmatrix}
\]

\NewDocumentCommand{\MDa}{}{\begin{pNiceArray}[margin]{ccc|cccc}
        0 & 1 & 0 & 0 & 0 & 0 & 0 \\
        0 & 0 & 0 & 0 & 0 & 0 & 0 \\
        0 & 0 & 0 & 0 & 0 & 0 & 0 \\
        \midrule
        0 & 0 & 0 & 0 & 1 & 0 & 0 \\
        0 & 0 & 0 & 0 & 0 & 0 & 0 \\
        0 & 0 & 0 & 1 & 0 & 0 & 0 \\
        0 & 0 & 0 & 0 & 0 & 0 & 0
    \end{pNiceArray}}
\NewDocumentCommand{\MDb}{}{\begin{pNiceArray}[margin]{ccc|cccc}
        1 & 0 & 0 & 0 & 0 & 0 & 0 \\
        0 & 0 & 1 & 0 & 0 & 0 & 0 \\
        0 & 0 & 0 & 0 & 0 & 0 & 0 \\
        \midrule
        0 & 0 & 0 & 0 & 0 & 0 & 1 \\
        0 & 0 & 0 & 0 & 0 & 1 & 0 \\
        0 & 0 & 0 & 0 & 0 & 0 & 0 \\
        0 & 0 & 0 & 1 & 0 & 0 & 0
    \end{pNiceArray}}

Для синхронизации обхода составим набор блочно-диагональных матриц, каждая из которых~--- это прямая сумма двух матриц: $D_{0\_a} = R_{0\_a} \oplus G_{0\_a}$ и $D_{0\_a} = R_{0\_b} \oplus G_{0\_b}$.
\[
    D_{0\_a} = \MDa \quad
    D_{0\_b} = \MDb
\]

Пусть мы решаем частный случай задачи достижимости с несколькими стартовыми вершинами (multiple-source)~--- достижимость с одной стартовой вершиной (single-source).

Пусть единственной начальной вершиной в графе будет вершина $0$.

Теперь создадим вектор
$v = \begin{pNiceArray}[]{ccc|cccc}
        1 & 0 & 0 & 1 & 0 & 0 & 0
    \end{pNiceArray}$,
где в первой части стоит единица на месте начального состояния $0$ в автомате.
Во второй части содержится фронт обхода графа, на первом шаге это всегда множество стартовых вершин.
В данном случае единица стоит на месте единственной стартовой вершины~--- $0$.

Совершим один шаг в обходе графа и получим новый фронт обхода графа.
\begin{widepar}
    \begin{gather*}
        a: \begin{pNiceArray}[]{ccc|cccc}
            1 & 0 & 0 & 1 & 0 & 0 & 0
        \end{pNiceArray} \cdot \MDa =
        \begin{pNiceArray}[]{ccc|cccc}
            0 & 1 & 0 & 0 & 1 & 0 & 0
        \end{pNiceArray}
        \\
        b: \begin{pNiceArray}[]{ccc|cccc}
            1 & 0 & 0 & 1 & 0 & 0 & 0
        \end{pNiceArray} \cdot \MDb  =
        \begin{pNiceArray}[]{ccc|cccc}
            1 & 0 & 0 & 0 & 0 & 0 & 1
        \end{pNiceArray}
    \end{gather*}
\end{widepar}

Сложим два полученных вектора, чтобы получить новый фронт обхода графа:
\begin{widepar}
    \[
        \begin{pNiceArray}[]{ccc|cccc}
            0 & 1 & 0 & 0 & 1 & 0 & 0
        \end{pNiceArray} +
        \begin{pNiceArray}[]{ccc|cccc}
            1 & 0 & 0 & 0 & 0 & 0 & 1
        \end{pNiceArray} =
        \begin{pNiceArray}[]{ccc|cccc}
            1 & 1 & 0 & 0 & 1 & 0 & 1
        \end{pNiceArray}.
    \]
\end{widepar}

То есть в наш фронт
$\begin{pmatrix}
        0 & 1 & 0 & 1
    \end{pmatrix}$
попали вершины 1 и 3 соотвественно.
А именно, мы совершили следующие переходы в графе и автомате.

\begin{minipage}{0.45\textwidth}
    \begin{center}
        \begin{tikzpicture}[node distance=2cm,shorten >=1pt,on grid,auto]
            \node[state, red] (q_0)   {$1$};
            \node[state] (q_1) [above=of q_0] {$2$};
            \node[state] (q_2) [right=of $(q_0)!0.5!(q_1)$] {$0$};
            \node[state, red] (q_3) [right=of q_2] {$3$};
            \path[->, red]
            (q_2) edge  node[pos=0.7] {a} (q_0)
            (q_2) edge[bend left]  node[above] {b} (q_3);
            \path[->]
            (q_0) edge  node {b} (q_1)
            (q_1) edge  node[pos=0.3] {a} (q_2)
            (q_3) edge[bend left]  node {b} (q_2);
        \end{tikzpicture}
    \end{center}
\end{minipage}
\begin{minipage}{0.45\textwidth}
    \begin{center}
        \begin{tikzpicture}[shorten >=1pt,on grid,auto]
            \node[state, draw=none]      (q_3) at (0,0)  {$ $}; % empty node for alignment
            \node[state, initial, red]   (q_0) at (0,1)  {$0$};
            \node[state, red]            (q_1) at (2,1)  {$1$};
            \node[state, accepting]      (q_2) at (4,1)  {$2$};
            \path[->, red]
            (q_0) edge  node {$a$} (q_1);
            \path[->]
            (q_1) edge  node {$b$} (q_2);
            \draw (q_0) edge[loop above, red]  node {$b$} (q_0);
        \end{tikzpicture}
    \end{center}
\end{minipage}

Совершим еще один шаг алгоритма.
Теперь вектор $v$, на который мы умножаем матрицы, имеет следующий вид
$\begin{pNiceArray}[]{ccc|cccc}
        1 & 1 & 0 & 0 & 1 & 0 & 1
    \end{pNiceArray}$.
\begin{widepar}
    \begin{gather*}
        a: \begin{pNiceArray}[]{ccc|cccc}
            1 & 1 & 0 & 0 & 1 & 0 & 1
        \end{pNiceArray} \cdot \MDa =
        \begin{pNiceArray}[]{ccc|cccc}
            0 & 1 & 0 & 0 & 0 & 0 & 0
        \end{pNiceArray} \displaybreak[0] \\ % FIXME: remove \displaybreak[0]
        b: \begin{pNiceArray}[]{ccc|cccc}
            1 & 1 & 0 & 0 & 1 & 0 & 1
        \end{pNiceArray} \cdot \MDb =
        \begin{pNiceArray}[]{ccc|cccc}
            1 & 0 & 1 & 1 & 0 & 1 & 0
        \end{pNiceArray} \\
        \begin{pNiceArray}[]{ccc|cccc}
            0 & 1 & 0 & 0 & 0 & 0 & 0
        \end{pNiceArray} +
        \begin{pNiceArray}[]{ccc|cccc}
            1 & 0 & 1 & 1 & 0 & 1 & 0
        \end{pNiceArray} =
        \begin{pNiceArray}[]{ccc|cccc}
            1 & 1 & 1 & 1 & 0 & 1 & 0
        \end{pNiceArray}
    \end{gather*}
\end{widepar}
То есть в наш фронт
$\begin{pmatrix}
        1 & 0 & 1 & 0
    \end{pmatrix}$
попали вершины 0 и 2 соотвественно.
Мы совершили следующие переходы в графе и автомате.

\begin{minipage}{0.45\textwidth}
    \begin{center}
        \begin{tikzpicture}[node distance=2cm,shorten >=1pt,on grid,auto]
            \node[state, gray] (q_0)   {$1$};
            \node[state, teal] (q_1) [above=of q_0] {$2$};
            \node[state, teal] (q_2) [right=of $(q_0)!0.5!(q_1)$] {$0$};
            \node[state, gray] (q_3) [right=of q_2] {$3$};
            \path[->, gray]
            (q_2) edge  node[pos=0.7] {a} (q_0)
            (q_2) edge[bend left]  node[above] {b} (q_3);
            \path[->, red]
            (q_0) edge  node {b} (q_1)
            (q_3) edge[bend left]  node {b} (q_2);
            \path[->]
            (q_1) edge  node[pos=0.3] {a} (q_2);
        \end{tikzpicture}
    \end{center}
\end{minipage}
\begin{minipage}{0.45\textwidth}
    \begin{center}
        \begin{tikzpicture}[shorten >=1pt,on grid,auto]
            \node[state, draw=none]            (q_3) at (0,0)  {$ $}; % empty node for alignment
            \node[state, initial, red]         (q_0) at (0,1)  {$0$};
            \node[state, gray]                 (q_1) at (2,1)  {$1$};
            \node[state, accepting, teal]      (q_2) at (4,1)  {$2$};
            \path[->, gray]
            (q_0) edge  node {$a$} (q_1);
            \path[->, red]
            (q_1) edge  node {$b$} (q_2);
            \draw (q_0) edge[loop above, red]  node {$b$} (q_0);
        \end{tikzpicture}
    \end{center}
\end{minipage}

При этом, можно заметить, что мы достигли конечной вершины в автомате.
Последний элемент левой части результирующего вектора
$\begin{pNiceArray}[]{ccc|cccc}
        1 & 1 & \textcolor{red}{1} & 1 & 0 & 1 & 0
    \end{pNiceArray}$
отвечает за состояние 2, которое является конечным.
А значит, обход необходимо остановить, и текущие вершины фронта обхода графа записать в ответ.

Таким образом, вершины графа 0 и 2 являются ответом.
Однако вершина 0~--- лишняя.
Регулярное выражение $b^*ab$ не подразумевает, что вершина 0 в графе может быть достигнута.
Она могла бы быть достигнута по пустой строке в случае, если бы регулярное выражение имело вид $b^*$ или по строке $aba$ в случае, если бы регулярное выражение имело вид $b^*aba$.

Это произошло, потому что в векторе $v$ должна кодироваться информация о паре~--- вершине графа и состоянии автомата.
Достигнув вершины 0, мы оказались в конечном состоянии автомата, которое было получено с помощью другой вершины~--- вершины 2.

Эту проблему можно решить, закодировав информацию о каждой такой паре в несколько векторов $v$, и ограничив левую часть вектора $v$ таким образом, чтобы в ней всегда была лишь одна единица.

Тогда мы получим, что вектор $v$ вида
$\begin{pNiceArray}[]{ccc|cccc}
        1 & 0 & 0 & 1 & 0 & 1 & 0
    \end{pNiceArray}$
будет хранить информацию о парах $(0, 0)$ и $(0, 2)$, где первый элемент пары~--- состояние автомата, а второй~--- вершина графа.

Аналогично, вектор $v$ вида
$\begin{pNiceArray}[]{ccc|cccc}
        0 & 1 & 0 & 0 & 1 & 1 & 0
    \end{pNiceArray}$
кодирует информацию о парах $(1, 1)$ и $(1, 2)$.
Вектор $v$ вида
$\begin{pNiceArray}[]{ccc|cccc}
        0 & 0 & 1 & 0 & 0 & 1 & 1
    \end{pNiceArray}$
кодирует информацию о парах $(2, 2)$ и $(2, 3)$.

Таким образом, мы будем понимать, в каком состоянии автомата мы находимся для каждой из вершин фронта обхода графа.

Рассмотрим, как это применяется в разработанном алгоритме, который представлен далее.

Предлагается \enquote{расклеить} $v$ в матрицу $M$, состоящую из трех векторов, добавив два вектора $\begin{pNiceArray}[]{ccc|cccc}
        0 & 1 & 0 & 0 & 0 & 0 & 0
    \end{pNiceArray}$
и $\begin{pNiceArray}[]{ccc|cccc}
        0 & 0 & 1 & 0 & 0 & 0 & 0
    \end{pNiceArray}$.
Во второй части этих векторов стоят нули, так как
$\begin{pmatrix}
        0 & 1 & 0
    \end{pmatrix}$
и
$\begin{pmatrix}
        0 & 0 & 1
    \end{pmatrix}$
кодируют состояния автомата 1 и 2, которые не являются начальными.
\NewDocumentCommand{\MMf}{}{\begin{pNiceArray}[]{ccc|cccc}
        1 & 0 & 0 & 1 & 0 & 0 & 0 \\
        0 & 1 & 0 & 0 & 0 & 0 & 0 \\
        0 & 0 & 1 & 0 & 0 & 0 & 0
    \end{pNiceArray}}
\[M = \MMf\]

И совершать обход тем же самым образом, но сохранив с помощью матрицы $M$ дополнительную информацию о парах (состояние, вершина).
\begin{widepar}
    \begin{gather*}
        a: \MMf \cdot \MDa = \begin{pNiceArray}[]{ccc|cccc}
            0 & 1 & 0 & 0 & 1 & 0 & 0 \\
            0 & 0 & 0 & 0 & 0 & 0 & 0 \\
            0 & 0 & 0 & 0 & 0 & 0 & 0
        \end{pNiceArray} \\
        b: \MMf \cdot \MDb = \begin{pNiceArray}[]{ccc|cccc}
            1 & 0 & 0 & 0 & 0 & 0 & 1 \\
            0 & 0 & 1 & 0 & 0 & 0 & 0 \\
            0 & 0 & 0 & 0 & 0 & 0 & 0
        \end{pNiceArray}
    \end{gather*}
\end{widepar}

Для того, чтобы левая часть матрицы $M$ всегда оставалось единичной, нужно трансформировать в ней строчки особым образом.
Для этого нужно складывать только те вектора в правой части матрицы $M$, у которых в левой части единицы стоят на одинаковых позициях.
После чего переставлять строчки в $M$ так, чтобы левая часть матрицы $M$ принимала единичный вид.
Вектора с пустой левой частью нас при этом не интересуют.

Тогда правая часть матрицы $M$ будет кодировать текущий фронт обхода графа.

В нашем примере матрица $M$ для следующего шага обхода выглядит следующим образом.
\NewDocumentCommand{\MMs}{}{\begin{pNiceArray}[]{ccc|cccc}
        1 & 0 & 0 & 0 & 0 & 0 & 1 \\
        0 & 1 & 0 & 0 & 1 & 0 & 0 \\
        0 & 0 & 1 & 0 & 0 & 0 & 0
    \end{pNiceArray}}
\[M = \MMs\]

Видно, что во фронт обхода графа попали вершины 1 и 3.
В вершину 1 мы попали в состоянии 1, в вершину 3~--- в состоянии 0.

Совершаются следующие переходы в графе и автомате.

\begin{minipage}{0.45\textwidth}
    \begin{center}
        \begin{tikzpicture}[node distance=2cm,shorten >=1pt,on grid,auto]
            \node[state, red] (q_0)   {$1$};
            \node[state] (q_1) [above=of q_0] {$2$};
            \node[state] (q_2) [right=of $(q_0)!0.5!(q_1)$] {$0$};
            \node[state, red] (q_3) [right=of q_2] {$3$};
            \path[->, red]
            (q_2) edge  node[pos=0.7] {a} (q_0)
            (q_2) edge[bend left]  node[above] {b} (q_3);
            \path[->]
            (q_0) edge  node {b} (q_1)
            (q_1) edge  node[pos=0.3] {a} (q_2)
            (q_3) edge[bend left]  node {b} (q_2);
        \end{tikzpicture}
    \end{center}
\end{minipage}
\begin{minipage}{0.45\textwidth}
    \begin{center}
        \begin{tikzpicture}[shorten >=1pt,on grid,auto]
            \node[state, draw=none]      (q_3) at (0,0)  {$ $}; % empty node for alignment
            \node[state, initial, red]   (q_0) at (0,1)  {$0$};
            \node[state, red]            (q_1) at (2,1)  {$1$};
            \node[state, accepting]      (q_2) at (4,1)  {$2$};
            \path[->, red]
            (q_0) edge  node {$a$} (q_1);
            \path[->]
            (q_1) edge  node {$b$} (q_2);
            \draw (q_0) edge[loop above, red]  node {$b$} (q_0);
        \end{tikzpicture}
    \end{center}
\end{minipage}

Сделаем еще один шаг алгоритма и придем к конечному состоянию в автомате.
\NewDocumentCommand{\MMt}{}{\begin{pNiceArray}[]{ccc|cccc}
        1 & 0 & 0 & 1 & 0 & 0 & 0 \\
        0 & 1 & 0 & 0 & 0 & 0 & 0 \\
        0 & 0 & 1 & 0 & 0 & 1 & 0
    \end{pNiceArray}}

\begin{widepar}
    \begin{gather*}
        \MMs \cdot \MDa = \begin{pNiceArray}[]{ccc|cccc}
            0 & 1 & 0 & 0 & 0 & 0 & 0 \\
            0 & 0 & 0 & 0 & 0 & 0 & 0 \\
            0 & 0 & 0 & 0 & 0 & 0 & 0
        \end{pNiceArray} \\
        \MMs \cdot \MDb = \begin{pNiceArray}[]{ccc|cccc}
            1 & 0 & 0 & 1 & 0 & 0 & 0 \\
            0 & 0 & 1 & 0 & 0 & 1 & 0 \\
            0 & 0 & 0 & 0 & 0 & 0 & 0
        \end{pNiceArray} \\
        M = \MMt
    \end{gather*}
\end{widepar}

Видно, что мы достигли вершины 2 графа в конечном состоянии 2 автомата.
При этом вершина 0 графа так же достигнута, как и в наивном варианте алгоритма, но теперь известно, что это происходит в состоянии 0 автомата.

\begin{minipage}{0.45\textwidth}
    \begin{center}
        \begin{tikzpicture}[node distance=2cm,shorten >=1pt,on grid,auto]
            \node[state, gray] (q_0)   {$1$};
            \node[state, teal] (q_1) [above=of q_0] {$2$};
            \node[state, red] (q_2) [right=of $(q_0)!0.5!(q_1)$] {$0$};
            \node[state, gray] (q_3) [right=of q_2] {$3$};
            \path[->, gray]
            (q_2) edge  node[pos=0.7] {a} (q_0)
            (q_2) edge[bend left]  node[above] {b} (q_3);
            \path[->, red]
            (q_0) edge  node {b} (q_1)
            (q_3) edge[bend left]  node {b} (q_2);
            \path[->]
            (q_1) edge  node[pos=0.3] {a} (q_2);
        \end{tikzpicture}
    \end{center}
\end{minipage}
\begin{minipage}{0.45\textwidth}
    \begin{center}
        \begin{tikzpicture}[shorten >=1pt,on grid,auto]
            \node[state, draw=none]            (q_3) at (0,0)  {$ $}; % empty node for alignment
            \node[state, initial, red]         (q_0) at (0,1)  {$0$};
            \node[state, gray]                 (q_1) at (2,1)  {$1$};
            \node[state, accepting, teal]      (q_2) at (4,1)  {$2$};
            \path[->, gray]
            (q_0) edge  node {$a$} (q_1);
            \path[->, red]
            (q_1) edge  node {$b$} (q_2);
            \draw (q_0) edge[loop above, red]  node {$b$} (q_0);
        \end{tikzpicture}
    \end{center}
\end{minipage}

Таким образом, в ответ попадает вершина 2.

Перейдем к формальному описанию алгоритма.

\subsection{Формальное описание алгоритма}

Алгоритм принимает на вход граф $\mscrG$, детерминированный конечный автомат $\mscrR$, описывающий регулярную грамматику, и множество начальных вершин $V_{\mathrm{src}}$ графа.

Граф $\mscrG$ и автомат $\mscrR$ можно представить в виде булевых матриц смежности.
Так, в виде словаря для каждой метки графа заводится булева матрица смежности, на месте $(i, j)$ ячейки которой стоит 1, если $i$ и $j$ вершины графа соединены ребром данной метки.
Такая же операция проводится для автомата грамматики $\mscrR$.

Далее, мы оперируем с двумя словарями, где ключом является символ метки ребра графа или символ алфавита автомата, а значением~--- соответствующая им булевая матрица.

Для каждого символа из пересечения этих множеств строится матрица $\mfrakD$, как прямая сумма булевых матриц.
То есть, строится матрица $\mfrakD = \mathrm{Bool}_{\mscrR_a} \oplus \mathrm{Bool}_{\mscrG_a}$, которая определяется как
\[
    \mfrakD = \begin{pmatrix}
        \mathrm{Bool}_{\mscrR_a} & 0                        \\
        0                        & \mathrm{Bool}_{\mscrG_a}
    \end{pmatrix},
\]
где $\mscrR_{a}$ и $\mscrG_{a}$ матрицы смежности соответствующих символов автомата грамматики $\mscrR$ и графа $\mscrG$ для символа $a \in A_\mscrR \cap A_\mscrG$, $A_\mscrR \cap A_\mscrG$~--- пересечение алфавитов.
Такая конструкция позволяет синхронизировать алгоритм обхода в ширину одновременно для графа и грамматики.

Далее вводится матрица $M$, хранящая информацию о фронте обхода графа.
\marginnote{TODO: Надо подумать над обозначениями. Почему $Id$, а не $E$? Может быть вообще надо эти обозначения вводить сразу в примере.}
Она нужна для выделения множества пройденных вершин и не допускает зацикливание алгоритма.
\[
    M^{k \times (k + n)} = \begin{pNiceArray}[]{c|c}
        Id_k & Matrix_{k \times n }
    \end{pNiceArray},
\]
где $Id_k$~--- единичная матрица размера $k$, $k$~--- количество вершин в автомате $\mscrR$, $Matrix_{k \times n }$~--- матрица, хранящая в себе маску пройденных вершин в автомате графа, $n$~--- количество вершин в графе $\mscrG$.

\subsection{Выходные данные}

На выходе строится множество $\mscrP$ пар вершин $(v, w)$ графа $\mscrG$ таких, что вершина $w$ достижима из множества начальных вершин, при этом $v \in V_{\mathrm{src}}$, $w \not\in V_{\mathrm{src}}$.
Это множество представляется в виде матрицы размера $|V| \times |V|$, где $(i,j)$ ячейка содержит 1, если пара вершин с индексами $(i, j) \in \mscrP$.

\subsection{Процесс обхода графа}

Алгоритм обхода заключается в последовательном умножении матрицы $M$ текущего фронта на матрицу $\mfrakD$.
В результате чего, находится матрица $M'$ содержащая информацию о вершинах, достижимых на следующем шаге.
Далее, с помощью операций перестановки и сложения векторов $M'$ преобразуется к виду матрицы $M$ и присваивается ей.
Итерации продолжаются пока $M'$ содержит новые вершины, не содержащиеся в $M$. На листинге~\ref{BFSRPQ1} представлен этот алгоритм.

% \begin{algorithm}[t]
%     \caption{Алгоритм достижимости в графе с регулярными ограничениями на основе поиска в ширину, выраженный с помощью операций матричного умножения}\label{BFSRPQ1}
%     \begin{algorithmic}[1]
%         \Procedure{BFSBasedRPQ}{$\mscrR=\langle Q, \Sigma, P, F, q \rangle,\mscrG=\langle V, E, L \rangle, V_{\mathrm{src}}$}
%         \State $\mscrP\gets~${Матрица смежности графа}
%         \State $\mfrakD\gets \mathrm{Bool}_\mscrR \bigoplus \mathrm{Bool}_\mscrG$\Comment{Построение матриц $\mfrakD$}
%         \State $M\gets CreateMasks(|Q|,|V|)$ \Comment{Построение матрицы $M$}
%         \State $M'\gets SetStartVerts(M, V_{\mathrm{src}})$  \Comment{Заполнение нач. вершин}

%         \While{Матрица~$M$~меняется}{}
%         \State $M\gets M'\langle\neg M\rangle$\Comment{Применение комплементарной маски}
%         \ForAll{$a\in (\Sigma \cap L)$}
%         \State $M'\gets M~$any.pair$~\mfrakD$
%         \Comment{Матр. умножение в полукольце}
%         \State $M'\gets TransformRows(M')$\label{TransformRows}
%         \Comment{Приведение $M'$ к виду $M$}
%         \EndFor
%         \State {$Matrix\gets extractRightSubMatrix(M')$}
%         \State $V\gets Matrix.reduceVector()$ \Comment{Сложение по столбцам}
%         \For{$k \in 0\dots|V_{\mathrm{src}}|-1$}
%         \State $W\gets\mscrP.getRow(k)$
%         \State $\mscrP.setRow(k, V+W)$
%         \EndFor
%         \EndWhile
%         \State \textbf{return} $\mscrP$
%         \EndProcedure
%     \end{algorithmic}
% \end{algorithm}

В алгоритме~\ref{BFSRPQ1}, в~\ref{TransformRows} строке происходит трансформация строчек в матрице $M'$.
Это делается для того, чтобы представить полученную во время обхода матрицу $M'$, содержащую новый фронт, в виде матрицы $M$.
Для этого требуется так переставить строчки $M'$, чтобы она содержала корректные по своему определению значения.
То есть, имела единицы на главной диагонали, а все остальные значения в первых $k$ столбцах были нулями.
Подробнее эта процедура описана в листинге~\ref{AlgoTransformRows}.

% \begin{algorithm}[H]
%     \caption{Алгоритм трансформации строчек}\label{AlgoTransformRows}
%     \begin{algorithmic}[1]
%         \Procedure{TransformRows}{$M$}
%         \State{$T \gets extractLeftSubMatrix(M)$}
%         \State{$Ix, Iy \gets$ итераторы по индексам ненулевых элементов $T$}
%         \For{$i \in 0\dots|Iy|$}
%         \State{$R\gets M.getRow(Ix[i])$}
%         \State{$M'.setRow(Iy[i], R + M'.getRow(Iy[i]))$}
%         \EndFor
%         \EndProcedure
%     \end{algorithmic}
% \end{algorithm}

\subsection{Модификации алгоритма}

Рассмотрим $V_{\mathrm{src}}$~--- множество начальных вершин, состоящее из $r$ элементов.
Для каждой начальной вершины $V_{\mathrm{src}}^i \in V_{\mathrm{src}}$ отметим соответствующие индексы в матрице $M$ единицами, получив матрицу $M(V_{\mathrm{src}}^i)$, и построим матрицу $\mfrakM$ следующим образом.
\[
    \mfrakM^{(k*r) \times (k + n)} = \begin{pmatrix}
        M(V_{\mathrm{src}}^1) \\
        M(V_{\mathrm{src}}^2) \\
        M(\dots)              \\
        M(V_{\mathrm{src}}^r)
    \end{pmatrix}
\]

Матрица $\mfrakM$ собирается из множества матриц $M(V_{\mathrm{src}}^i)$ и позволяет хранить информацию о том, из какой начальной вершины достигаются новые вершины во время обхода.

% \begin{algorithm}[t]
%     \caption{Модификация алгоритма для поиска конкретной исходной вершины}\label{BFSRPQ2}
%     \begin{algorithmic}[1]
%         \Procedure{BFSBasedRPQ}{$\mscrR=\langle Q, \Sigma, P, F, q \rangle,\mscrG=\langle V, E, L \rangle, V_{\mathrm{src}}$}
%         \State $\mscrP\gets~${Матрица смежности графа}
%         \State $\mfrakD\gets \mathrm{Bool}_\mscrR \bigoplus \mathrm{Bool}_\mscrG$
%         \State $\mfrakM\gets CreateMasks(|Q|,|V|)$
%         \State $\mfrakM'\gets SetStartVerts(\mfrakM, V_{\mathrm{src}})$

%         \While{Матрица~$\mfrakM$~меняется}{}
%         \State $\mfrakM\gets \mfrakM'\langle\neg\mfrakM\rangle$
%         \ForAll{$a\in (\Sigma \cap L)$}
%         \State $\mfrakM'\gets \mfrakM~$any.pair$~\mfrakD$
%         \ForAll{$M \in \mfrakM'$}
%         \State $M\gets TransformRows(M)$
%         \EndFor
%         \EndFor
%         \ForAll{$M_k \in \mfrakM'$}
%         \State $Matrix\gets extractSubMatrix(M)$
%         \State $V\gets Matrix.reduceVector()$
%         \State $W\gets\mscrP.getRow(k)$
%         \State $\mscrP.setRow(k, V+W)$
%         \EndFor
%         \EndWhile
%         \State \textbf{return} $\mscrP$
%         \EndProcedure
%     \end{algorithmic}
% \end{algorithm}

В листинге~\ref{BFSRPQ2} представлен модифицированный алгоритм.
Основное его отличие заключается в том, что для каждой достижимой вершины находится конкретная исходная вершина, из которой начинался обход.

Таким образом, алгоритмы~\ref{BFSRPQ1} и~\ref{BFSRPQ2} решают сформулированные в пункте~\ref{sec:3.3} задачи достижимости.
