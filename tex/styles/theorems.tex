
\usepackage{thmtools}

%%% theorem-like envs
\theoremstyle{definition}

\declaretheoremstyle[spaceabove=0.5\topsep,
    spacebelow=0.5\topsep,
    headfont=\bfseries\sffamily,
    bodyfont=\normalfont,
    headpunct=.,
    postheadspace=5pt plus 1pt minus 1pt]{myStyle}
\declaretheoremstyle[spacebelow=\topsep,
    headfont=\bfseries\sffamily,
    bodyfont=\normalfont,
    headpunct=.,
    postheadspace=5pt plus 1pt minus 1pt,]{myStyleWithFrame}
\declaretheoremstyle[spacebelow=\topsep,
    headfont=\itshape\sffamily,
    bodyfont=\normalfont,
    headpunct=.,
    postheadspace=5pt plus 1pt minus 1pt,
    qed=\blacksquare]{myProofStyleWithFrame}

\tcbuselibrary{breakable, skins}
\tcbset{shield externalize}
\tcbset{boxrule=0pt,
    sharp corners,
    borderline west={0.3mm}{0pt}{black},
    frame hidden,
    enhanced,
    interior hidden,
    left=2mm,
    top = 1mm,
    bottom = 1mm,
    right = 0.5mm
}

\tcolorboxenvironment{theorem}{}
% \tcolorboxenvironment{theorem*}{}
% \tcolorboxenvironment{axiom}{}
% \tcolorboxenvironment{assertion}{}
\tcolorboxenvironment{lemma}{}
% \tcolorboxenvironment{proposition}{}
% \tcolorboxenvironment{corollary}{}
\tcolorboxenvironment{definition}{}
% \tcolorboxenvironment{proofReplace}{toprule=0mm,bottomrule=0mm,rightrule=0mm, colback=white, breakable }

\declaretheorem[name=Теорема, numberwithin=chapter, style=myStyleWithFrame]{theorem}
% \declaretheorem[name=Теорема, numbered=no, style=myStyleWithFrame]{theorem*}
% \declaretheorem[name=Аксиома, sibling=theorem, style=myStyleWithFrame]{axiom}
% \declaretheorem[name=Преположение, sibling=theorem, style=myStyleWithFrame]{assertion}
\declaretheorem[name=Лемма, sibling=theorem, style=myStyleWithFrame]{lemma}
% \declaretheorem[name=Предложение, sibling=theorem, style=myStyleWithFrame]{proposition}
% \declaretheorem[name=Следствие, numberwithin=theorem, style=myStyleWithFrame]{corollary}

\declaretheorem[name=Определение, numberwithin=chapter, style=myStyleWithFrame]{definition}
% \declaretheorem[name=Свойство, numberwithin=chapter, style=myStyle]{property}
% \declaretheorem[name=Свойства, numbered=no, style=myStyle]{propertylist}

\declaretheorem[name=Пример, numberwithin=chapter, style=myStyle]{example}
\declaretheorem[name=Замечание, numbered=no, style=myStyle]{remark}

\declaretheorem[name=Доказательство, numbered=no, style=myProofStyleWithFrame]{proofReplace}
\renewenvironment{proof}[1][\proofname]{\begin{proofReplace}}{\end{proofReplace}}
% \declaretheorem[name=Доказательство, numbered=no, style=myProofStyleWithFrame]{longProof}

\declaretheorem[name={Набросок доказательства}, numbered=no, style=myProofStyleWithFrame]{proofSketch}
