\documentclass[
    fontsize=10pt,
    a4paper,
    twoside=false,
    parskip=false,
]{kaobook}

\usepackage{polyglossia} % Локализация документа --- переносы и всё такое
\setmainlanguage{russian}
\setotherlanguage{english}

\usepackage[autostyle]{csquotes} % Правильные кавычки в зависимости от языка
\usepackage{microtype} % Полезные типографические ништячки

\usepackage{xurl} % Разрешить переносить URL на любой букве
\usepackage[noheader]{gitver}
\usepackage{caption}
\usepackage{subcaption}

\NewDocumentCommand{\email}{m}{\href{mailto:#1}{#1}} % Кликабельный email

%% : с чуть более узки расстоянием по бокам.
%% Определено в pseudo, потому пока что закомментировано
 \NewDocumentCommand \rng { } {
 \nolinebreak
 \mathinner { : }
 \nolinebreak
 }

\usepackage{tikz}
\usetikzlibrary{arrows.meta}
\usetikzlibrary{external}
\usetikzlibrary{positioning}
\usetikzlibrary{shapes.geometric}
\usetikzlibrary{automata}
\usetikzlibrary{decorations.pathmorphing}
\usetikzlibrary{backgrounds}
\usetikzlibrary{calc}
\usetikzlibrary{arrows}
\usetikzlibrary{fit}


\tikzsetexternalprefix{figures/externalized/}

% Required in RegularLanguages
\tikzset{snake it/.style={decorate, decoration=snake}}
\tikzset{
    side by side/.style 2 args={
            line width=2pt,
            #1,
            postaction={
                    clip,postaction={draw,#2}
                }
        }
}

% \usetikzlibrary{fit, calc}

\usepackage{amsmath, amsfonts, amssymb, amsthm, mathtools} % Advanced math tools.
\usepackage{nicematrix}

% \setmathfont[range={\doubleplus}, Scale=MatchLowercase]{Asana Math}
\NewDocumentCommand{\derives}{O{*}}{\xRightarrow[]{#1}}


\usepackage{thmtools}

%%% theorem-like envs
\theoremstyle{definition}

\declaretheoremstyle[spaceabove=0.5\topsep,
    spacebelow=0.5\topsep,
    headfont=\bfseries\sffamily,
    bodyfont=\normalfont,
    headpunct=.,
    postheadspace=5pt plus 1pt minus 1pt]{myStyle}
\declaretheoremstyle[spacebelow=\topsep,
    headfont=\bfseries\sffamily,
    bodyfont=\normalfont,
    headpunct=.,
    postheadspace=5pt plus 1pt minus 1pt,]{myStyleWithFrame}
\declaretheoremstyle[spacebelow=\topsep,
    headfont=\itshape\sffamily,
    bodyfont=\normalfont,
    headpunct=.,
    postheadspace=5pt plus 1pt minus 1pt,
    qed=\blacksquare]{myProofStyleWithFrame}

\tcbuselibrary{breakable, skins}
\tcbset{shield externalize}
\tcbset{boxrule=0pt,
    sharp corners,
    borderline west={0.3mm}{0pt}{black},
    frame hidden,
    enhanced,
    interior hidden,
    left=2mm,
    top = 1mm,
    bottom = 1mm,
    right = 0.5mm
}

\tcolorboxenvironment{theorem}{}
% \tcolorboxenvironment{theorem*}{}
% \tcolorboxenvironment{axiom}{}
% \tcolorboxenvironment{assertion}{}
\tcolorboxenvironment{lemma}{}
% \tcolorboxenvironment{proposition}{}
% \tcolorboxenvironment{corollary}{}
\tcolorboxenvironment{definition}{}
% \tcolorboxenvironment{proofReplace}{toprule=0mm,bottomrule=0mm,rightrule=0mm, colback=white, breakable }

\declaretheorem[name=Теорема, numberwithin=chapter, style=myStyleWithFrame]{theorem}
% \declaretheorem[name=Теорема, numbered=no, style=myStyleWithFrame]{theorem*}
% \declaretheorem[name=Аксиома, sibling=theorem, style=myStyleWithFrame]{axiom}
% \declaretheorem[name=Преположение, sibling=theorem, style=myStyleWithFrame]{assertion}
\declaretheorem[name=Лемма, sibling=theorem, style=myStyleWithFrame]{lemma}
% \declaretheorem[name=Предложение, sibling=theorem, style=myStyleWithFrame]{proposition}
% \declaretheorem[name=Следствие, numberwithin=theorem, style=myStyleWithFrame]{corollary}

\declaretheorem[name=Определение, numberwithin=chapter, style=myStyleWithFrame]{definition}
% \declaretheorem[name=Свойство, numberwithin=chapter, style=myStyle]{property}
% \declaretheorem[name=Свойства, numbered=no, style=myStyle]{propertylist}

\declaretheorem[name=Пример, numberwithin=chapter, style=myStyle]{example}
\declaretheorem[name=Замечание, numbered=no, style=myStyle]{remark}

\declaretheorem[name=Доказательство, numbered=no, style=myProofStyleWithFrame]{proofReplace}
\renewenvironment{proof}[1][\proofname]{\begin{proofReplace}}{\end{proofReplace}}
% \declaretheorem[name=Доказательство, numbered=no, style=myProofStyleWithFrame]{longProof}

\declaretheorem[name={Набросок доказательства}, numbered=no, style=myProofStyleWithFrame]{proofSketch}


\usepackage{kaobiblio} % Обертка для biblatex, позволяет печатать сноску сбоку
\addbibresource{FormalLanguageConstrainedReachabilityLectureNotes.bib}

\usepackage[]{pseudo}
\newtcbtheorem{algorithm}{Листинг}{pseudo/booktabs, float, floatplacement=h, separator sign={.}}{algo}

\usepackage{caption}
\usepackage{subcaption}

\pghyphenation[]{russian}{%
тео-ре-ти-ко-мно-жест-вен-ных
}

\tikzexternalize

\title{О достижимости с ограничениями в терминах формальных языков}
\author{Семён Григорьев}
\date{\gitVer{}}

\begin{document}

\frontmatter

\maketitle
\tableofcontents

\addchap{Список авторов}

\begin{description}[style=nextline]
      \item[Семён Григорьев]
            Санкт-Петербургский государственный университет, Университетская набережная, 7/9, Санкт-Петербург, 199034, Россия \\
            \email{s.v.grigoriev@spbu.ru}\\
            JetBrains Research, Приморский проспект 68-70, здание 1, Санкт-Петербург, 197374, Россия \\
            \email{semyon.grigorev@jetbrains.com}
      \item[Екатерина Вербицкая]
            JetBrains Research, Приморский проспект 68-70, здание 1, Санкт-Петербург, 197374, Россия \\
            \email{ekaterina.verbitskaya@jetbrains.com}
      \item[Дмитрий Кутленков]
            Санкт-Петербургский государственный университет, Университетская набережная, 7/9, Санкт-Петербург, 199034, Россия \\
            \email{kutlenkov.dmitri@gmail.com}
      \item[]
\end{description}

Полный список людей, внёсших свой вклад в данную работу, можно посмотреть на страничке проекта:
\url{https://github.com/FormalLanguageConstrainedPathQuerying/FormalLanguageConstrainedReachability-LectureNotes}


\mainmatter
\setchapterstyle{kao}

\input{Introduction}
\input{LinearAlgebra}
\input{GraphTheoryIntro}
\input{FormalLanguageTheoryIntro}
\setchapterpreamble[u]{\margintoc}
\chapter{Регулярные языки}
\tikzsetfigurename{RegularLanguages_}

В данном разделе мы обсудим регулярные языки~--- класс, лежащий на самом нижнем уровне иерархии Хомского.
Будут рассмотрены основные способы задания таких языков: \emph{регулярные выражения}, \emph{конечные автоматы}, \emph{лево(право)линейные грамматики}.
Обсудим основные свойства регулярных языков, такие как замкнутость относительно различных операций, а также различные свойства соответствующих автоматов и грамматик.

\section{Регулярные выражения}

Регулярные выражения~--- один из классических способов задать регулярный язык%
\sidenote{
    Замечание для программистов.
    Важно понимать, что речь идёт о формальной конструкции, а не о том, что называется регулярными выражениями в различных языках программирования или библиотеках, где под названием \enquote{регулярные выражения} могут скрываться конструкции, существенно более выразительные, чем обсуждаемые здесь.
}.
Основывается этот способ на предложении синтаксиса для описания \emph{регулярных множеств}%
\sidenote{Помним, что язык~--- это множество слов.}.

\begin{definition}[Регулярное множество]
    Регулярное множество (над алфавитом $\Sigma$) это:
    \begin{itemize}
        \item $\varnothing$;
        \item $\{\varepsilon\}$;
        \item $\{t\}$, $t \in \Sigma$;
        \item $R_1 \cup R_2$, где $R_1$ и $R_2$~--- регулярные множества;
        \item $R_1 \cdot R_2$, где $R_1$ и $R_2$~--- регулярные множества;
        \item $R^*$, где $R$~--- регулярное множество.
    \end{itemize}
\end{definition}

Для того, чтобы описывать такие множества, удобно пользоваться \emph{регулярными выражениями}.

\begin{definition}[Регулярное выражение]
    Регулярное выражение (над алфавитом $\Sigma$) это:
    \begin{itemize}
        \item $\varnothing$;
        \item $\varepsilon$;
        \item $t$, $t \in \Sigma$;
        \item $R_1 \mid R_2$, где $R_1$ и $R_2$~--- регулярные выражения;
        \item $R_1 \cdot R_2$, где $R_1$ и $R_2$~--- регулярные выражения;
        \item $R^*$, где $R$~--- регулярное выражение;
        \item $(R)$, где $R$~--- регулярное выражение.
    \end{itemize}
\end{definition}

Отметим несколько важных с прикладной точки зрения моментов.
Во-первых, часто используется расширенный синтаксис, в который добавляются конструкции не увеличивающие выразительную силу, но упрощающие запись.
Например, встречаются следующие расширения%
\sidenote{
    Существуют и другие, однако их мы не будем использовать и, соответственно, рассматривать.
    Читатель может вспомнить, что называется регулярными выражениями в его любимом языке программирования и попробовать самостоятельно выразить имеющиеся там конструкции через базовые.}.
\begin{itemize}
    \item $R? = R \mid \varepsilon$, где $R$~--- регулярное выражение.
    \item $R^+ = R \cdot R^*$, где $R$~--- регулярное выражение.
\end{itemize}

Во-вторых, конструкции $\varnothing$ и $\varepsilon$ используются крайне редко, особенно в случае расширенного синтаксиса, так как часто выражение, эквивалентное использующему данные конструкции, часто более компактно записывается с использованием расширенного синтаксиса.
В-третьих, оператор конкатенации часто опускается%
\sidenote{Как и знак умножения во многих математических записях.}.

Рассмотрим несколько примеров регулярных выражений.
\begin{example}
    Регулярное выражение $a$ задаёт регулярное множество $\{a\}$ и, соответственно, язык из единственного слова $a$.
\end{example}


\begin{example}
    Регулярное выражение $ab$ задаёт регулярное множество $\{ab\}$ и, соответственно, язык из единственного слова $ab$.
\end{example}


\begin{example}
    Регулярное выражение $a^*$ задаёт регулярное множество $$R = \bigcup_{i=0}^{\infty}{a^i} = \{\varepsilon, a, aa, aaa, \ldots \}$$ и, соответственно, бесконечный язык, содержащий для любого неотрицательного целого $n$ цепочку из символов $a$ длины $n$.
\end{example}

\begin{example}
    $a^*b$
\end{example}

\begin{example}
    $(a\mid b)^*$
\end{example}

\begin{example}
    $(ab)^*c?$
\end{example}

\section{Конечные автоматы}

\emph{Конечный автомат}~--- вычислительная машина, которая имеет конечный набор состояний и может совершать переходы между ними, читая входные данные.
Важно отметить, что ни какой дополнительной памяти классический конечный автомат не имеет%
\sidenote{Существуют автоматы с константной памятью, регистрами} и не производит дополнительных действий%
\sidenote{Автоматы с записью на ленту, и т.д.}.

\begin{definition}[Недетерминированный конечный автомат]
    \label{def:NondeterminicticFiniteAutomata}
    \emph{Недетерминированный конечный автомат} (НКА)~--- это пятёрка $M = \langle Q, Q_S, Q_F, \delta, \Sigma \rangle$, где
    \begin{itemize}
        \item $Q$~--- конечное множество состояний;
        \item $Q_S \subseteq= Q$~--- множество стартовых состояний;
        \item $Q_F \subseteq Q$~--- множество финальных состояний;
        \item $\delta \subseteq Q \times (\Sigma \cup \varepsilon) \times 2^Q$~--- функция переходов, а $\varepsilon \notin \Sigma$;
        \item $\Sigma$~--- конечный алфавит.
    \end{itemize}
\end{definition}

Так как нас интересуют конечные автоматы в контексте языков, то будем говорить, что на ленте автомата записано какое-то слово (или строка).
Иными словами, будем говорить, что автомат принимает на вход слово или строку.

Процесс вычислений, проделываемых конечным автоматом, удобно описывать в терминах переходов между \emph{конфигурациями}.

\begin{definition}[Конфигурация]
    Конфигурация $c$ конечного автомата $M = \langle Q, Q_S, Q_F, \delta, \Sigma \rangle$~--- это пара $(q, w)$, где $q\in Q$~--- это текущее состояние автомата, а $w \in \Sigma^*$~--- непросмотренная часть входной строки.
\end{definition}

\begin{definition}[Переход в НКА]
    Будем говорить, что автомат $M = \langle Q, Q_S, Q_F, \delta, \Sigma \rangle$ может перейти из конфигурации $c_1 = (q_1, w_1)$ в конфигурацию $c_2 = (q_2, w_2)$, если
    \[c_2 \in \{(q_2,w_2) \mid w_1 = aw_2, (q_1,a, q_2) \in \delta\} \cup \{(q_2,w_1) \mid (q_1, \varepsilon, q_2) \in \delta\}.\]
    Обозначать этот факт будем как $c_1 \to c_2$.
    Также будем считать, что на множестве конфигураций задано отношение перехода $(\to):(Q \times \Sigma^*)\times(Q \times \Sigma^*)$.
    
\end{definition}

\begin{definition}
    Транзитивное замыкание отношения перехода на конфигурациях будем обозначать следующим образом: $$ c_1 \to^* c_2. $$
    Альтернативно, в случае если $c_1 \to^* c_2$, будем говорить, что конфигурация $c_2$ \textit{достижима} из конфигурации $c_1$.
\end{definition}

Для удобства работы с недетерминированными автоматами расширим это отношение на множество конфигураций.

\begin{definition}
Будем говорить, что автомат может перейти из множества конфигураций $C_1$ в множество конфигураций $C_2$, если 
$$C_2 = \bigcup_{c_1 \in C_1} \{c_2 \mid c_1 \to c_2 \}.$$

Обозначать этот факт будем как  $C_1 \Rightarrow C_2 $.
\end{definition}

\begin{definition}
Транзитивное замыкание отношения перехода на множествах конфигураций будем обозначать следующим образом: $$ C_1 \Rightarrow^* C_2. $$
\end{definition}

Для описания работы автомата $M = \langle Q, Q_S, Q_F, \delta, \Sigma \rangle$ нам понадобятся следующие выделенные типы конфигураций.
\begin{itemize}
    \item Стартовая конфигурация $c_s = (q_s,w)$, $q_s \in Q_S$, $w$ --- цепочка, которая подаётся на вход автомату. 
    Для недетерминированного автомата естественно задать множество стартовых конфигурация $C_S = \bigcup_{q_s \in Q_S} (q_s,w)$.
    \item Финальная (принимающая) конфигурация $c_f = (q_f,\varepsilon)$, $q_f \in Q_F$.
\end{itemize}

Таким образом, работу автомата можно описать как последовательность переходов между множествами конфигураций. 
Работа начинается с множества стартовых конфигураций и завершается в следующих двух случаях.
\begin{enumerate}
    \item Очередное множество конфигураций содержит финальную конфигурацию:
    $$c_f \in C_S \text{ или } C_S \Rightarrow^* C_i, c_f \in C_i.$$ В этом случае говорят, что автомат \textit{принимает} входную строку.
    \item Очередное множество конфигураций пусто:
    $$C_0 = C_S \Rightarrow^* C_i \Rightarrow^* \varnothing, \text{ для любого } i: c_f \notin C_i.$$ 
    В этом случае говорят, что автомат \textit{не принимает} или \textit{отвергает} входную строку.
\end{enumerate} 

\begin{definition}
    Язык задаваемый автоматом $$\{w \mid \}$$
\end{definition}

Так как конфигурация полностью описывает состояние процесса вычислений, то не надо обрабатывать одну и ту же конфигурацию несколько раз. 
Это поможет при написании реального интерпретатора. 
Будем отслеживать уже посещённые (обработанные) конфигурации\footnote{Техника, аналогичная той, что применяется в обходах графов (обход в ширину, обход в глубину) для того, чтобы избежать повторного посещения вершин и, как следствие, зацикливания обхода. Более того, она типична для алгоритмов с рабочим множеством.}. 

\begin{example}
    Пример интерпретации конечного автомата.
\end{example}

\begin{definition}[Детерминированный конечный автомат]
    \label{def:DeterminicticFiniteAutomata}
    \emph{Детерминированный конечный автомат} (ДКА, Deterministic Finite Automata, DFA)~--- это пятёрка $M = \langle Q, q_S, Q_F, \delta, \Sigma \rangle$, где
    \begin{itemize}
        \item $Q$~--- конечное множество состояний;
        \item $q_S \in Q$~--- стартовое состояние;
        \item $Q_F \subseteq Q$~--- множество финальных состояний;
        \item $\delta \subseteq Q \times \Sigma \times Q$~--- функция переходов\footnote{Частично определённая.};
        \item $\Sigma$~--- конечный алфавит.
    \end{itemize}
\end{definition}

Отличие --- функция переходов. Нет переходов по $\varepsilon$ и из любого состояния не более одного перехода по символу.
Ещё стартовое состояние одно.

Заметим, что функцию переходов можно представить разными способами В зависимости от того, как именно представлена функция переходов: список троек, матрица, граф.

\begin{example}
    Пример КА.
    % \begin{tikzpicture}

    % \end{tikzpicture}
\end{example}

\begin{example}
    Пример интерпретации конечного автомата.
\end{example}

\section{Производные для регулярных языков}

Предложены в~\cite{Brzozowski1964}

По мотивам~\cite{OWENS_REPPY_TURON_2009}

\begin{itemize}
    \item $\partial_t(\varepsilon) = \varnothing$
    \item $\partial_t(\varnothing) = \varnothing$
    \item $\partial_t(x) = $
    \item $\partial_t(R_1 \cdot R_2) = \partial_t(R_1) \cdot (R_2) \mid $
    \item $\partial_t(R_1 \mid R_2) = \partial_t(R_1) \mid \partial_t(R_2) $
    \item $\partial_t(R^*) = $\footnote{Интересное упражнение --- показать это, расписав по определению звезду Клини.}
\end{itemize}

Проверка на пустоту (часто isNull).

\begin{itemize}
    \item $IsNull(\varepsilon) = false$
    \item $\partial_t(\varnothing) = \varnothing$
    \item $\partial_t(x) = $
    \item $\partial_t(R_1 \cdot R_2) = \partial_t(R_1) \cdot (R_2) \mid $
    \item $\partial_t(R_1 \mid R_2) = \partial_t(R_1) \mid \partial_t(R_2) $
    \item $\partial_t(R^*) = $\footnote{Интересное упражнение --- показать это, расписав по определению звезду Клини.}
\end{itemize}

Проверка пустоты регулярного языка\footnote{!!!!}


\section{Построение конечного автомата по регулярному выражению}
 
На производных.

Примеры.

\section{Построение регулярного выражения по конечному автомату}

Регулярное выражение будем строить по недетерминированному автомату специального вида: потребуем, чтобы у него было ровно одно стартовое состояние и ровно одно финальное\footnote{Любой автомат легко привести к такому виду: добавить состояния и $\varepsilon$-переходы}. 

Будем в цикле выполнять последовательно две операции.
Первая: объединение параллельных рёбер.
Вторая: устранение вершины $v$. За один шаг можем устранить любую кроме стартовой или финальной.
Цикл повторяется до тех пор, пока в автомате не останется ровно два состояния: стартовое и финальное.

До объединения параллельных рёбер
\begin{tikzpicture}

\node[state] (q_0)          {$q_i$};
\node[state] (q_1) [right of = q_0]  {$q_j$};
\path[->]
  (q_0) edge[bend left, above]  node {$R_1$} (q_1)
  (q_0) edge[bend right, below]  node {$R_2$} (q_1)
  ;
\end{tikzpicture}

После объединения параллельных рёбер.

\begin{tikzpicture}

    \node[state] (q_0)          {$q_i$};
    \node[state] (q_1) [right of = q_0]  {$q_j$};
    \path[->]
      (q_0) edge[above]  node {$R_1 \mid R_2$} (q_1)
      ;
    \end{tikzpicture}
    

\begin{tikzpicture}

\begin{scope}[node distance=10mm and 10mm]
   \node[state] (p_0)          {$p_0$};
   \node[text width=0.3cm]  (p_1) [below of = p_0] {$\vdots$}; 
   \node[state] (p_2) [below of = p_1]  {$p_i$};
   \node[text width=0.3cm] (p_3) [below of = p_2]  {$\vdots$};
\end{scope}

\node[state] (v_0) [right of = p_2] {$v$};

\begin{scope}[node distance=10mm and 10mm]
    \node[state] (q_2) [right of = v_0] {$q_j$};
    \node[text width=0.3cm] (q_1) [above of = q_2]  {$\vdots$};
    \node[state] (q_0) [above of = q_1]  {$q_0$};
    \node[text width=0.3cm] (q_3) [below of = q_2]  {$\vdots$};
\end{scope}

\path[->]
  (p_0) edge[above]  node {$R_{p_0}$} (v_0)
  (p_2) edge[below]  node {$R_{p_i}$} (v_0)
  (v_0) edge[right]  node {$R_{q_0}$} (q_0)
  (v_0) edge[above]  node {$R_{q_j}$} (q_2)
  (v_0) edge[loop above, above]  node {$R_v$} (v_0);
\end{tikzpicture}


\begin{tikzpicture}

    \begin{scope}[node distance=10mm and 10mm]
       \node[state] (p_0)          {$p_0$};
       \node[text width=0.3cm]  (p_1) [below of = p_0] {$\vdots$}; 
       \node[state] (p_2) [below of = p_1]  {$p_i$};
       \node[text width=0.3cm] (p_3) [below of = p_2]  {$\vdots$};
    \end{scope}
    
    \node[text width=0.3cm] (v_0) [right of = p_2] {};
    
    \begin{scope}[node distance=10mm and 40mm]
        \node[state] (q_2) [right of = v_0] {$q_j$};
        \node[text width=0.3cm] (q_1) [above of = q_2]  {$\vdots$};
        \node[state] (q_0) [above of = q_1]  {$q_0$};
        \node[text width=0.3cm] (q_3) [below of = q_2]  {$\vdots$};    
    \end{scope}
    
    \path[->]
      (p_0) edge[bend left, above]  node {$R_{p_0} R_v^* R_{q_0}$} (q_0)
      (p_0) edge[bend left, left]  node {$R_{p_0} R_v^* R_{q_j}$} (q_2)
      (p_2) edge[bend right, left]  node {$R_{p_i} R_v^* R_{q_0}$} (q_0)
      (p_2) edge[bend right, below]  node {$R_{p_0} R_v^* R_{q_0}$} (q_2);
    \end{tikzpicture}

$p_i v q_j$

$p_i \xrightarrow{R_{p_i}} v$
$v \xrightarrow{R_{q_j}} q_i$
$v \xrightarrow{R_v} v$
$p_i \xrightarrow{R_{p_i} \cdot R_v^* \cdot R_{q_j}} q_j$

По финальному автомату с двумя состояниями построим регулярное выражение, которое би будет ответом.

\begin{tikzpicture}
    \node[isosceles triangle,
    isosceles triangle apex angle=60,
    draw=none,fill=none,
    minimum size=2cm] (T60) at (3,0){};

\node[state, initial] (q_0)          {$0$};
\node[state, accepting] (q_1) [right of = q_0]  {$1$};
\path[->]
  (q_0) edge[bend left, above]  node {$R_2$} (q_1)
  (q_1) edge[bend left, below]  node {$R_4$} (q_0)
  (q_1) edge[loop right, right]  node {$R_3$} (q_1)
  (q_0) edge[loop above, above]  node {$R_1$} (q_0);
\end{tikzpicture}

$R_1^* \cdot (R_2 \cdot R_3^* \cdot R_4 \cdot R_1^*)^* \cdot R_2 \cdot R_3^*$



Примеры.



\section{Лево(право)линейные грамматики}

Наложив некоторые ограничения на внешний вид правил грамматики можно получить грамматики, задающие регулярные языки.

\begin{definition}[Леволинейная грамматика]
    Грамматика $G=\langle \Sigma, N, P, S \rangle$ называется леволиненйной, если все её правила имеют вид
    \[N_i \to \alpha w,\]
    где $N_i \in N$, $\alpha \in \{\varepsilon\} \cup N$, $w \in \Sigma ^*$.
\end{definition}

\begin{definition}[Праволинейная грамматика]
    Грамматика $G=\langle \Sigma, N, P, S \rangle$ называется праволиненйной, если все её правила имеют вид
    \[N_i \to  w \alpha,\]
    где $N_i \in N$, $\alpha \in \{\varepsilon\} \cup N$, $w \in \Sigma ^*$.
\end{definition}

Ноам Хомский и Джордж Миллер в работе~\sidecite{chomsky1958finite} показали, что лево(право)линейные грамматики задают регулярные языки.
Приведём процедуры построения автомата по грамматике и наоборот, грамматики по автомату.

Пусть дан конечный автомат $M = \langle \Sigma, Q, q_s, Q_f, \delta \rangle$. По нему можно построить праволинейную грамматику $G=\langle \Sigma, N, S, P \rangle$, где
\begin{itemize}
    \item $N = Q$
    \item $P = \{ q_i \to t q_j \mid (q_i, t, q_j)\in \delta\} \cup \{ q_i \to \varepsilon \mid q_i \in Q_F\}$
    \item $S = q_s$
\end{itemize}

Аналогичным образом строится автомат по праволинейной грамматике.
Упростить процедуру можно если заранее привести правила к виду $N_i \to tN_j$, где $t\in \Sigma$, добавив необходимое количество новых нетерминалов:
правило вида $N_i \to twN_k$ преобразуется в два правила
\begin{align*}
    N_i & \to tN_l  \\
    N_l & \to wN_k,
\end{align*}
после чего аналогично преобразуется правило для $N_l$.

Пример построения грамматики по автомату.

Автомат по грамматике.

\section{Лемма о накачке}

Лемма о накачке для регулярных языков позволяет проверить, что заданный язык не является регулярным.

\begin{lemma}
    Пусть $L$~--- регулярный язык над алфавитом $\Sigma$, тогда существует такое $n$, что для любого слова $\omega \in L$, $|\omega| \geq n$ найдутся слова $x,y,z\in \Sigma^*$, для которых верно: $xyz = \omega, y\neq \varepsilon,|xy|\leq n$ и для любого $k \geq 0$  $xy^kz \in L$.
\end{lemma}

\begin{proofSketch}
    \begin{enumerate}
        \item Так как язык регулярный, то для него можно построить конечный автомат $M = \langle Q, q_s,Q_f, \delta, \Sigma \rangle$.
              В том числе, минимальный по количеству состояний.
        \item В качестве $n$ возьмём $|Q| + 1$.
        \item Легко заметить, что для любой цепочки $w \in L, |w| > n$ путь в автомате, соответствующий принятию данной цепочки, будет содержать хотя бы один цикл.
              Действительно, в ориентированном графе с $k$ вершинами (а именно таким является автомат по построению) максимальная длина пути без повторных посещений вершин (соответственно, без циклов) не больше $k - 1$.
        \item Выберем любой цикл. Он будет задавать искомые цепочки $x, y$ и $z$ так, как представлено на рисунке~\ref{fig:reg_lang_pumping_lemma}.
              Заметим, что вход в цикл и выход из него в общем случае могут не совпадать, что даёт несколько вариантов разбиения пути на части, и на рисунке представлен лишь один из возможных.
              \qedhere
    \end{enumerate}
\end{proofSketch}

\begin{figure}
    \caption{Иллюстрация идеи доказательства леммы о накачке для регулярных языков: любой путь в графе, длина которого достаточно большая, может быть разбит на три части из леммы ($x$~--- красный подпуть, $y$~--- синий, $z$~--- зелёный), а многократный проход по циклу $y$ позволяет \enquote{накачать} слово.}
    \label{fig:reg_lang_pumping_lemma}
    \begin{center}
        \begin{tikzpicture}[->]
            \node[state, initial] (q1) {$q_1$};
            \node[state, right = 2 of q1] (q2) {$q_2$};
            \node[state, accepting, right = 2 of q2] (q3) {$q_3$};
            \node[state, above = 2 of q2] (q4) {$q_4$};
            \draw (q1) edge[above, snake it] node{} (q2)
            (q2) edge[bend right, snake it, side by side={red}{blue}] node{} (q4)
            (q4) edge[bend right, snake it, color=blue] node[left]{$y$} (q2)
            (q4) edge[above, snake it, color=green] node[right]{$z$} (q3)
            (q1) edge[above, snake it, color=red] node{$x$} (q2)
            ;
        \end{tikzpicture}
    \end{center}

\end{figure}


\section{Замкнутость регулярных языков относительно теоретико-множественных операций}

\begin{theorem}
    Регулярные языки замкнуты относительно перечисленных ниже операций.
    \begin{enumerate}
        \item Пересечение
        \item Дополнение
        \item Обращение
        \item Разность
    \end{enumerate}
\end{theorem}

Линейная алгебра для работы с регулярными языками: пересечение, замыкание.

Построение пересечения через тензорное произведение автоматов.

Идея доказательства, что мы построили именно пересечение.

Пересечение через синхронный обход в ширину.

%\section{Вопросы и задачи}
%
%Построить базу.
%
%Научиться выполнять запросы через линейку.

\setchapterpreamble[u]{\margintoc}
\chapter{Контекстно-свободные языки и грамматики}
\label{CFG}
\tikzsetfigurename{CFG_}

Контекстно-свободные языки, являясь строгим расширением регулярных, находят своё применение в различных задачах анализа программного кода, биоинформатики~\sidecite{!!!}, и других задачах анализа данных.

\section{Основные определения}

\begin{definition}[Контекстно-свободная грамматика]
    \emph{Контекстно-свободная грамматика}~--- это четвёрка вида $\langle \Sigma, N, P, S \rangle$, где
    \begin{itemize}
        \item $\Sigma$~--- это терминальный алфавит;
        \item $N$~--- это нетерминальный алфавит;
        \item $P$~--- это множество правил (продукций), таких что каждая продукция имеет вид $N_i \to \alpha$, где $N_i \in N$ и $\alpha \in \{\Sigma \cup N\}^* \cup {\varepsilon}$;
        \item $S$~--- стартовый нетерминал.
              Отметим, что $\Sigma \cap N = \varnothing$.
    \end{itemize}
\end{definition}

\begin{example}
    \label{ex:binary_cfg}
    Грамматика, задающая язык целых чисел в двоичной записи без лидирующих нулей: $G = \langle \{0, 1, -\}, \{S, N, A\}, P, S \rangle$, где $P$ определено следующим образом:
    \begin{align*}
        S & \rightarrow 0 \mid N \mid - N              \\
        N & \rightarrow 1 A                            \\
        A & \rightarrow 0 A \mid 1 A  \mid \varepsilon
    \end{align*}
\end{example}

При спецификации грамматики часто опускают множество терминалов и нетерминалов, оставляя только множество правил.
При этом нетерминалы часто обозначаются прописными латинскими буквами, терминалы~--- строчными, а стартовый нетерминал обозначается буквой $S$.
Мы будем следовать этим обозначениям, если не указано иное.

\begin{definition}[Отношение непосредственной выводимости]
    \label{def derivability in CFG}
    \emph{Отношение непосредственной выводимости}. Мы говорим, что последовательность терминалов и нетерминалов $\gamma \beta \delta$ \emph{непосредственно выводится из} $\gamma \alpha \delta$ \emph{при помощи правила} $\alpha \rightarrow \beta$ ($\gamma \alpha \delta \derives[] \gamma \beta \delta$), если
    \begin{itemize}
        \item $\alpha \rightarrow \beta \in P$,
        \item $\gamma, \delta \in \{\Sigma \cup N\}^* \cup {\varepsilon}$.
    \end{itemize}
\end{definition}

\begin{definition}[Рефлексивно-транзитивное замыкание отношения]
    \emph{Рефлексивно-транзитивное замыкание отношения}~--- это наименьшее рефлексивное и транзитивное отношение, содержащее исходное.
\end{definition}

\begin{definition}[Отношение выводимости]
    \emph{Отношение выводимости} является рефлексивно-транзитивным замыканием отношения непосредственной выводимости; обозначается $\derives$.
    \begin{itemize}
        \item $\alpha \derives \beta$ означает $\exists \gamma_0, \dots, \gamma_k \in \{\Sigma \cup N\}^* \cup {\varepsilon}$:
              \[ \alpha \derives[] \gamma_0 \derives[] \gamma_1 \derives[] \dots \derives[] \gamma_{k-1} \derives[] \gamma_{k} \derives[] \beta;\]
        \item Транзитивность: $\forall \alpha, \beta, \gamma \in \{\Sigma \cup N\}^* \cup {\varepsilon}$: $\alpha \derives \beta$, $\beta \derives \gamma \derives[] \alpha \derives \gamma$;
        \item Рефлексивность: $\forall \alpha \in \{\Sigma \cup N\}^* \cup {\varepsilon}$: $\alpha \derives \alpha$;
        \item $\alpha \derives \beta$~--- $\alpha$ выводится из $\beta$;
        \item $\alpha \derives[k] \beta$~--- $\alpha$ выводится из $\beta$ за $k$ шагов;
        \item $\alpha \derives[+] \beta$~--- при выводе использовалось хотя бы одно правило грамматики.
    \end{itemize}
\end{definition}

\begin{example}
    Пример вывода цепочки $-1101$ в грамматике из примера~\ref{ex:binary_cfg}:
    \[
        S \derives[] - N \derives[] - 1 A \derives[] - 1 1 A \derives - 1 1 0 1 A \derives[] - 1 1 0 1
    \]
\end{example}

\begin{definition}[Вывод слова в грамматике]
    Слово $\omega \in \Sigma^*$ \emph{выводимо в грамматике} $\langle \Sigma, N, P, S \rangle$, если существует некоторый вывод этого слова из начального нетерминала $S \derives \omega$.

\end{definition}

\begin{definition}[Левосторонний вывод]
    \emph{Левосторонний вывод}. На каждом шаге вывода заменяется самый левый нетерминал.
\end{definition}

\begin{definition}[Правосторонний вывод]
    \emph{Правосторонний вывод}. На каждом шаге вывода заменяется самый правый нетерминал.
\end{definition}

\begin{example}
    Приведем пример левостороннего вывода цепочки $cbaa$ в грамматике:
    \begin{align*}
        S & \rightarrow A A \mid s          \\
        A & \rightarrow A A \mid B b \mid a \\
        B & \rightarrow c \mid d
    \end{align*}
    Жирным выделен нетерминал, заменяемый на каждом шагу вывода.
    \[ \mathbf{S} \derives[] \mathbf{A} A \derives[] \mathbf{B} b A \derives[] c b \mathbf{A} \derives[] c b \mathbf{A} A \derives[] c b a \mathbf{A} \derives[] c b a a \]
\end{example}

Аналогично левостороннему можно определить правосторонний вывод.

\begin{definition}[Язык]
    \emph{Язык, задаваемый грамматикой}~--- множество строк, выводимых в грамматике $L(G) = \{ \omega \in \Sigma^* \mid S \derives \omega \}$.
\end{definition}

\begin{definition}[Эквивалентные грамматики]
    Грамматики $G_1$ и $G_2$ называются \emph{эквивалентными}, если они задают один и тот же язык: $L(G_1) = L(G_2)$
\end{definition}

\begin{example}  Пример эквивалентных грамматик для языка целых чисел в двоичной системе счисления.

    \begin{tabular}{p{0.4\textwidth} | p{0.5\textwidth}}
        \[
            \begin{aligned}
                \Sigma & = \{ 0, 1, - \}                            \\
                N      & = \{ S, N, A \}                            \\~\\
                S      & \rightarrow 0 \mid N \mid - N              \\
                N      & \rightarrow 1 A                            \\
                A      & \rightarrow 0 A \mid 1 A  \mid \varepsilon \\
            \end{aligned}
        \]
         &
        \[
            \begin{aligned}
                \Sigma & = \{ 0, 1, - \}                            \\
                N      & = \{ S, A \}                               \\~\\
                S      & \rightarrow 0 \mid 1 A  \mid - 1 A         \\
                A      & \rightarrow 0 A \mid 1 A  \mid \varepsilon \\
            \end{aligned}
        \]
    \end{tabular}

\end{example}


\begin{definition}[Неоднозначная грамматика]
    \emph{Неоднозначная грамматика}~--- грамматика, в которой существует 2 и более левосторонних (правосторонних) выводов для одного слова.
\end{definition}

\begin{example}
    Неоднозначная грамматика для правильных скобочных последовательностей:
    \[
        S \to (S) \mid S S \mid \varepsilon
    \]
    Два различных левосторонних вывода строки $()()()$:
    \begin{gather*}
        S \derives[] S S \derives [] (S) S \derives[] () S \derives[] () S S \derives[] () (S) S \derives[] () () S \derives[] () () (S) \derives[] () () ()\\
        S \derives[] S S \derives[] S S S \derives[] (S) S S \derives[] () S S \derives[] () (S) S \derives[] () () S \derives[] () () (S) \derives[] () () ()
    \end{gather*}
\end{example}

\begin{definition}[Однозначная грамматика]
    \emph{Однозначная грамматика}~--- грамматика, в которой существует не более одного левостороннего (правостороннего) вывода для каждого слова.
\end{definition}

\begin{example}
    Однозначная грамматика для правильных скобочных последовательностей
    \[
        S \to (S)S \mid \varepsilon
    \]
\end{example}

\begin{definition}[Существенно неоднозначный язык]
    \emph{Существенно неоднозначные языки}~--- языки, для которых невозможно построить однозначную грамматику.
\end{definition}

\begin{example}
    Пример существенно неоднозначного языка
    \[\{a^n b^n c^m \mid n, m \in \BbbZ\} \cup \{a^n b^m c^m \mid n,m \in \BbbZ\}\]
\end{example}

\section{Рекурсивные автоматы и сети}

Рекурсивный автомат (Recursive state machine, RSM)~\sidecite{10.1145/1075382.1075387} или сеть~--- это представление контекстно-свободных грамматик, обобщающее конечные автоматы.
В нашей работе мы будем придерживаться термина \emph{рекурсивный автомат}.
Классическое определение рекурсивного автомата выглядит следующим образом.

\begin{definition}
    \emph{Рекурсивный автомат}~--- это кортеж $\mathcal{R} = \langle \mathcal{N},\Sigma,B,B_S,Q,Q_S\rangle$ where
    \begin{itemize}
       \item $N$~--- множество нетерминальных символов, 
       \item $\Sigma$~--- множество терминальных символов, 
       \item $Q$~--- множество состояний рекурсивного автомата, 
       \item $Q_S$~--- множество стартовых состояний всех \emph{блоков}, 
       \item $B=\{B_{N_i} \mid N_i \in \mathcal{N}\}$~--- множество блоков, где каждый \\ $B_{N_i} = \langle Q_{N_i}, q_S, Q_F^{N_i}, \delta \rangle$~--- это детерминированный конечный автомат, называемый \emph{блоком}.
             Здесь:  
            \begin{itemize}
                \item $Q_{N_i} \subseteq Q$~--- множество состояний блока, 
                \item $q_S \in Q_{N_i} \cap Q_S$~--- стартовое состояние блока, 
                \item $Q_F^{N_i} \subseteq Q_{N_i}$~--- множество финальных состояний блока, 
                \item $\delta \subseteq Q_{N_i} \times  (\Sigma\cup Q_S) \times Q_{N_i}$~--- функция переходов блока.
            \end{itemize}
       \item $B_S \in B$~--- стартовый блок рекурсивного автомата.
    \end{itemize}
\end{definition}

Таким образом, рекурсивный автомат~--- это набор детерминированных конечных автоматов над алфавитом $\Sigma \cup Q_S$.
В некоторых случаях наб будет удобно рассматривать этот набор как одн конечный автомат.
Однако процесс работы рекурсивного автомата несколько отличается от работы конечного автомата, хотя и похож на него.
Основное отличие~--- необходимость использовать стек для обработки переходов, помеченных элементами из $Q_S$.

По аналогии с конечными автоматами, процесс работы рекурсивных автоматов достаточно естественно описывается в терминах переходов между \emph{конфигурациями}.

\begin{definition}
    \emph{Конфигурация} $C_{\mathcal{R}}$ рекурсивного автомата $\mathcal{R}=\langle \mathcal{N},\Sigma,B,B_S,Q,Q_S \rangle$ over the graph $D=\langle V,E,L \rangle$ is a tuple $(q,v,\mathcal{S})$ where 
    \begin{itemize}
        \item $q \in Q$ is a current state of RSM,
        \item $\mathcal{S}$ is the current stack, whose frames have one of two types: 
        \begin{itemize} 
            \item return addresses frame (elements of $Q$) to specify states to continue computation after the call is finished;
            \item parsing tree node to store fragments of a parsing tree,
        \end{itemize}
        \item $v \in V$ is the current vertex (current position in the input).
    \end{itemize}
\end{definition}

\begin{definition}\label{def:rsm_transition}
    A \emph{transition step} of the RSM specifies how to get new configurations of RSM, given the current configuration. $C_{\mathcal{R}} \vdash W$ denotes that $\mathcal{R}$ can go to each configuration in $W$ from the configuration $C_{\mathcal{R}}$.
    \begin{align*}
    (q,v,w_0::s::\mathcal{S})  \vdash & \{ (q',v',t::w_0::s::\mathcal{S}) \mid (q,t,q') \in \delta, (v,t,v') \in E\} \\
                       & \cup \{(s', v, q'::w_0::s::\mathcal{S}) \mid (q,s',q') \in \delta, s' \in Q_S \} \\
                       & \cup \{(s,v,\emph{Node}(N_i, w_0)::\mathcal{S}) \mid q \in Q_F^{N_i}\}
    \end{align*}
    where $w_0$ is a possibly empty sequence of terminals and nonterminal nodes. 
\end{definition}

To simplify the acceptance condition, we introduce the concept of an \textit{extended RSM}. 

\begin{definition}
    For the given RSM $\mathcal{R}=\langle \mathcal{N},\Sigma,B,B_S,Q,Q_S\rangle$,
    the \textit{extended RSM} 
    $$\mathcal{R}'=\langle \mathcal{N} \cup \{S'\},\Sigma \cup\{\$\},B \cup {B'_S},B'_S,Q \cup \{q_0',q_1',q_2'\},Q_S \cup \{q_0'\}\rangle$$
    is an RSM which defines the same language and is built from $\mathcal{R}$ by adding a new start box 
    $$B'_S = \langle \{q_0',q_1',q_2'\}, q_0', \{q_2'\}, \{(q_0',q_0,q_1'),(q_1',\$, q_2')\} \rangle$$ where 
    \begin{itemize}
        \item $q_0$ is a start state of $B_S$,
        \item \$ is a special symbol to mark the end of input, $\$ \notin \Sigma$,
        \item $q_i'$ are newly added states, $q_i'\notin Q$.
    \end{itemize}
\end{definition} 

Finally, for the given extended RSM $\mathcal{R}$ and the given graph $D$ we say that $v_n$ is reachable from $v_0$  w.r.t. $\mathcal{R}$ if $(q'_0,v_0,[]) \vdash^* C$ such that $(q'_1,v_n,[N_S]) \in C$, where $\vdash^*$ denotes zero or more transition steps, and $N_S$ is a node for the start nonterminal of the original (not extended) RSM. Additionally, $N_S$ represents a respective path.

Now we need a way to compute transitions and to build trees efficiently, avoiding recomputation and infinite cycles that are possible with the naive implementation.
Moreover, we need a compact representation of all paths of interest whose number can be infinite.

\subsection{Example}\label{section:example_of_rsm}

\begin{marginfigure}
    \begin{center}
        \resizebox{\marginparwidth}{!}{
            \begin{tikzpicture}[->,>=stealth']  

                \node[initial,state]   (F)              {$q_4$};
                \node[state]           (G) [right = of F] {$q_5$};
                \node[accepting,state] (H) [right = of G] {$q_6$};
                \node[draw=black, fit= (F) (G) (H), inner sep=0.25cm] (J) {};
                \node[below right] at (J.north west) {S'};

                \path (F) edge[below]              node {$q_0$} (G)
                    (G) edge[below]              node {$\$$} (H); 

                \node[initial,state] (A) [below = 2.6cm of F] {$q_0$};
                \node[state]         (B) [right = of A] {$q_1$};
                \node[state]         (D) [above right = of B] {$q_2$};
                \node[accepting,state]         (C) [right = of B] {$q_3$};
                \node [draw=black, fit= (A) (C) (D), inner sep=0.25cm] (E) {};
                \node[below right] at (E.north west) {S};

                \path (A) edge[below]              node {a} (B)
                      (B) edge[left]               node {$q_0$} (D)
                          edge[below]              node {b} (C)
                      (D) edge[right]              node {b} (C);
            \end{tikzpicture}
       }
    \end{center}    
    \caption{Расширенный рекурсивный автомат для грамматики $S \to a \ b \mid a \ S \ b$}
    \label{fig:example-rsm}
\end{marginfigure}

\begin{marginfigure}
    \begin{center}

        \begin{tikzpicture}[->,>=stealth',shorten >=1pt,auto,node distance=2.8cm, semithick]        

        \node[state] (A)                    {$v_0$};
        \node[state]         (B) [right of=A] {$v_1$};
        
        \path (A) edge  [loop left] node {a} (A)
                    edge  [bend left] node {b} (B)
                (B) edge  [bend left] node {b} (A);
        \end{tikzpicture}
    \end{center}
    \caption{Пример входного графа}
    \label{fig:input-graph}
\end{marginfigure}

In this section, we introduce a step-by-step example of the context-free constrained path querying using RSM and the naive computation strategy. 

Suppose that the input is a graph $D$ presented in figure~\ref{fig:input-graph} and the grammar  $G$ has two productions $S \to a \ b \mid a \ S \ b$. The start vertex is $v_0$ and our goal to find at least one path to each reachable vertex (w.r.t. $G$). The extended RSM for the given grammar is presented in figure~\ref{fig:example-rsm}. 

The initial configuration is $(q_4,v_0,[])$: we start from the initial state of the box for $S'$, the initial position in the graph is a start vertex $v_0$, the stack is empty. In each step, we apply rules from definition~\ref{def:rsm_transition} to compute new configurations. The sequence of transitions, presented below, allows us to find a path $$v_0 \xrightarrow{a} v_0 \xrightarrow{b} v_1$$
from $v_0$ to $v_1$ (step~\ref{eq:naive-rsm-step-res-1}) and a path 
$$v_0 \xrightarrow{a} v_0 \xrightarrow{a} v_0 \xrightarrow{b} v_1 \xrightarrow{b} v_0$$
from $v_0$ to itself (step~\ref{eq:naive-rsm-step-res-2}). 

\begin{align}
\setcounter{equation}{0}
(q_4,v_0,[])     \vdash & \{(q_0,v_0,[q_5])\} \\
(q_0,v_0,[q_5])  \vdash & \{(q_1,v_0,[a,q_5])\} \\
(q_1,v_0,[a,q_5])\vdash & \{(q_0,v_0,[q_2,a,q_5]) \nonumber\\ 
                        & , (q_3,v_1,[b,a,q_5])\} \\
(q_0,v_0,[q_2,a,q_5]) \vdash & \{(q_1,v_0,[a,q_2,a,q_5])\} \\
(q_3,v_1,[b,a,q_5])   \vdash & \boxed{\{(q_5,v_1,[S(a,b)])\}} \label{eq:naive-rsm-step-res-1}\\
(q_1,v_0,[a,q_2,a,q_5]) \vdash & \{(q_3,v_1,[b,a,q_2,a,q_5]) \nonumber \\
                               & , (q_0,v_0,[q_2,a,q_2,a,q_5])\} \label{eq:naive-rsm-step-6} \\
(q_3,v_1,[b,a,q_2,a,q_5]) \vdash & \{(q_2,v_1,[S(a,b),a,q_5])\} \\
(q_2,v_1,[S(a,b),a,q_5]) \vdash & \{(q_3,v_0,[b,S(a,b),a,q_5])\} \\
(q_3,v_0,[b,S(a,b),a,q_5]) \vdash & \boxed{\{(q_5,v_0,[S(a,S(a,b),b)])\}} \label{eq:naive-rsm-step-res-2}
\end{align}

Note that we have no conditions to stop computation. In our example, we can continue computation and get new paths between $v_0$ and $v_1$. Moreover, there is an infinite number of such paths. Additionally, the selection of the next step is not deterministic. One can choose the configuration $(q_0,v_0,[q_2,a,q_2,a,q_5])$ to continue computations after step~\ref{eq:naive-rsm-step-6} with chance to fall into an infinite cycle, but choosing $(q_3,v_1,[b,a,q_2,a,q_5])$ we get a new path.
Построим рекурсивный автомат для грамматики $G$:
\begin{align*}
    S & \to    a S b \\
    S & \to    a b
\end{align*}

\begin{figure}
    \caption{Пример рекурсивного автомата для грамматики $G$.}
    \label{input1}
    \begin{tikzpicture}[node distance=2.5cm,shorten >=1pt,on grid,auto]
        \node[state, initial] (q_0)   {$0 \{S\}$};
        \node[state] (q_1) [right=of q_0] {$1$};
        \node[state] (q_2) [right=of q_1] {$2$};
        \node[state, accepting] (q_3) [right=of q_2] {$3\{S\}$};
        \path[->]
        (q_0) edge  node {a} (q_1)
        (q_1) edge  node {S} (q_2)
        (q_2) edge  node {b} (q_3)
        (q_1) edge[bend left, above]  node {b} (q_3);
    \end{tikzpicture}
\end{figure}


Используем стандартные обозначения для стартовых и финальных состояний.
Дополнительно в стартовых и финальных состояниях укажем нетерминалы, для которых эти состояния стартовые/финальные.

В некоторых случаях рекурсивный автомат можно рассматривать как конечный автомат над смешанным алфавитом.
Именно такой взгляд мы будем использовать при изложении алгоритма.

Пример интерпретации конечного автомата.


\section{Дерево вывода}
\label{sect:DerivTree}

В некоторых случаях не достаточно знать порядок применения правил.
Необходимо структурное представление вывода цепочки в грамматике.
Таким представлением является \emph{дерево вывода}.

\begin{definition}[Дерево вывода]
    Деревом вывода цепочки $\omega$ в грамматике $G = \langle \Sigma, N, S, P \rangle$ называется корневое упорядоченное помеченное дерево, удовлетворяющее следующим свойствам.
    \begin{enumerate}
        \item Метка каждого внутреннего узла~--- нетерминал. При этом метка корня~--- стартовый нетерминал
        \item Метка каждого листа~--- терминал или $\varepsilon$.
        \item В дереве может существовать узел с меткой $N_i$ и сыновьями $M_j \dots M_k$ только тогда, когда в грамматике есть правило вида $N_i \to M_j \dots M_k$.
        \item Крона образует исходную цепочку $\omega$.
    \end{enumerate}
\end{definition}

\begin{example}
    Пусть дана грамматика
    \begin{equation}
        G = \langle \{a,b\}, \{S\}, S, \{S \to a \ S \ b \ S, S \to \varepsilon\} \rangle.\label{eq:grammar}
    \end{equation}
    Дерево вывода цепочки $abab$ в этой грамматике представлено на рисунке~\ref{fig:derivation_tree_example}. 
    
    \begin{marginfigure}
        \begin{center}
            \resizebox{\marginparwidth}{!}{\begin{tikzpicture}[sibling distance=3em,
  every node/.style = {shape=rectangle, rounded corners,
    draw, align=center}]
  \node {S}
    child { node {a} }
    child { node {S}
      child { node {$\varepsilon$}}
    }
    child { node {b} }
    child { node {S}
      child {node {a}}
      child { node {S}
        child { node {$\varepsilon$}}
      }
      child { node {b} }
      child { node {S}
          child {node {$\varepsilon$}}
        }
      };
\end{tikzpicture}}
        \end{center}
        \caption{Дерева вывода цепочки $abab$ в грамматике~\ref{eq:grammar}}
        \label{fig:derivation_tree_example}
    \end{marginfigure}
    
\end{example}

\begin{theorem}
    Пусть $G = \langle \Sigma, N, P, S \rangle$~--- КС-грамматика.
    Вывод $S \derives \alpha$, где $\alpha \in (N \cup \Sigma)^*, \alpha \neq \varepsilon$ существует $\Leftrightarrow$ существует дерево вывода в грамматике $G$ с кроной $\alpha$.
\end{theorem}

\section{Пустота КС-языка}

\begin{theorem}
    Существует алгоритм, определяющий, является ли язык, порождаемый КС грамматикой, пустым.
\end{theorem}

\begin{proof}
    Следующая лемма утверждает, что если в КС языке есть выводимое слово, то существует другое выводимое слово с деревом вывода не глубже количества нетерминалов грамматики.
    Для доказательства теоремы достаточно привести алгоритм, последовательно строящий все деревья глубины не больше количества нетерминалов грамматики, и проверяющий, являются ли такие деревья деревьями вывода.
    Если в результате работы алгоритма не удалось построить ни одного дерева, то грамматика порождает пустой язык.
\end{proof}

\begin{lemma}
    Если в данной грамматике выводится некоторая цепочка, то существует цепочка, дерево вывода которой не содержит ветвей длиннее $m$, где $m$~--- количество нетерминалов грамматики.
\end{lemma}

\begin{proof}
    Рассмотрим дерево вывода цепочки $\omega$. Если в нем есть 2 узла, соответствующих одному нетерминалу A, обозначим их $n_1$ и $n_2$.

    Предположим, $n_1$ расположен ближе к корню дерева, чем $n_2$.

    Вывод цепочки $\omega$ имеет следующий вид:
    \[S \derives \alpha A_{n_1} \beta \derives \alpha \omega_1 \beta; S \derives \alpha \gamma A_{n_2} \delta \beta \derives \alpha \gamma \omega_2 \delta \beta \derives \omega,\]
    при этом $\omega_2$ является подцепочкой $\omega_1$.

    Заменим в изначальном дереве узел $n_1$ на $n_2$. Полученное дерево является деревом вывода цепочки $\alpha \omega_2 \delta$.

    Повторяем процесс замены одинаковых нетерминалов до тех пор, пока в дереве не останутся только уникальные нетерминалы.

    В полученном дереве не может быть ветвей длины большей, чем $m$.

    По построению оно является деревом вывода.
\end{proof}


\section{Нормальная форма Хомского}
\label{section:CNF}

\begin{definition}[Нормальная форма Хомского]
    Контекстно-свободная грамматика $\langle \Sigma, N, P, S\rangle$ находится в \emph{Нормальной Форме Хомского}, если она содержит только правила следующего вида:
    \begin{itemize}
        \item $A \to B C$, где $A, B, C \in N$, а стартовый нетерминал $S$ не содержится в правой части правила.
        \item $A \to a$, где $A \in N$, $a \in \Sigma$.
        \item $S \to \varepsilon$: только из стартового нетерминала выводима пустая строка.
    \end{itemize}
\end{definition}

\begin{theorem}
    Любую КС грамматику можно преобразовать в НФХ.
\end{theorem}

\begin{proof}
    Алгоритм преобразования в НФХ состоит из следующих шагов:
    \begin{itemize}
        \item Замена неодиночных терминалов
        \item Удаление длинных правил
        \item Удаление $\varepsilon$-правил
        \item Удаление цепных правил
        \item Удаление бесполезных нетерминалов
    \end{itemize}
    То, что каждый из этих шагов преобразует грамматику к эквивалентной, при этом является алгоритмом, доказано в следующих леммах.
\end{proof}

\begin{lemma}
    Для любой КС-грамматики можно построить эквивалентную, которая не содержит правила с неодиночными терминалами.
\end{lemma}

\begin{proof}
    Каждое правило $A \to B_0 B_1 \dots B_k$, $k \geq 1$ заменить на множество правил, где $C_i$~--- новый нетерминал:
    \begin{align*}
        A   & \to C_0 C_1 \dots C_k \\
        C_0 & \to B_0               \\
        C_1 & \to B_1               \\
            & \dots                 \\
        C_k & \to B_k
    \end{align*}
\end{proof}

\begin{lemma}
    Для любой КС-грамматики можно построить эквивалентную, которая не содержит правил длины больше 2.
\end{lemma}

\begin{proof}
    Каждое правило $A \to B_0 B_1 \dots B_k$, $k \geq 2$ заменить на множество правил:
    \begin{align*}
        A       & \to B_0 C_0         \\
        C_0     & \to B_1 C_1         \\
                & \dots               \\
        C_{k-3} & \to B_{k-2} C_{k-2} \\
        C_{k-2} & \to B_{k-1} B_k
    \end{align*}
\end{proof}


\begin{lemma}
    Для любой КС-грамматики можно построить эквивалентную, не содержащую $\varepsilon$-правил.
\end{lemma}

\begin{proof}
    Рекурсивно определим $\varepsilon$-правила:
    \begin{itemize}
        \item $A \to \varepsilon$~--- $\varepsilon$-правило
        \item $A \to B_0 \dots B_k$~--- $\varepsilon$-правило, если $\forall i$: $B_i$~--- $\varepsilon$-правило.
    \end{itemize}
    Каждое правило $A \to B_0 B_1 \dots B_k$ заменяем на множество правил, где каждое $\varepsilon$-правило удалено во всех возможных комбинациях.
\end{proof}

\begin{lemma}
    Для любой КС-грамматики можно построить эквивалентную, не содержащую цепные правила.
\end{lemma}

\begin{proof}
    \emph{Цепное правило}~--- правило вида $A \to B\text{, где } A, B \in N$.
    \emph{Цепная пара}~--- упорядоченная пара $(A,B)$, в которой $A\derives B$, используя только цепные правила.

    Алгоритм:
    \begin{enumerate}
        \item Найти все цепные пары в грамматике $G$.
              Найти все цепные пары можно по индукции:
              Базис: $(A,A)$~--- цепная пара для любого нетерминала, так как $A\derives A$ за ноль шагов.
              Индукция: Если пара $(A,B_0)$~--- цепная, и есть правило $B_0 \to B_1$, то $(A,B_1)$~--- цепная пара.
        \item Для каждой цепной пары $(A,B)$ добавить в грамматику $G'$ все правила вида $A \to a$, где $B \to a$~--- нецепное правило из $G$.
        \item Удалить все цепные правила
    \end{enumerate}
    Пусть $G$~--- контекстно-свободная грамматика. $G'$~--- грамматика, полученная в результате применения алгоритма к $G$. Тогда $L(G)=L(G')$.
\end{proof}

\begin{definition}[Порождающий и непорождающий нетерминалы]
    Нетерминал $A$ называется \emph{порождающим}, если из него может быть выведена конечная терминальная цепочка.
    Иначе он называется \emph{непорождающим}.
\end{definition}

\begin{lemma}
    Можно удалить все бесполезные (непорождающие) нетерминалы.
\end{lemma}

\begin{proof}
    После удаления из грамматики правил, содержащих непорождающие нетерминалы, язык не изменится, так как непорождающие нетерминалы по определению не могли участвовать в выводе какого-либо слова.

    Алгоритм нахождения порождающих нетерминалов:
    \begin{enumerate}
        \item Множество порождающих нетерминалов пустое.
        \item Найти правила, не содержащие нетерминалов в правых частях и добавить нетерминалы, встречающихся в левых частях таких правил, в множество.
        \item Если найдено такое правило, что все нетерминалы, стоящие в его правой части, уже входят в множество, то добавить в множество нетерминалы, стоящие в его левой части.
        \item Повторить предыдущий шаг, если множество порождающих нетерминалов изменилось.
    \end{enumerate}
    В результате получаем множество всех порождающих нетерминалов грамматики, а все нетерминалы, не попавшие в него, являются непорождающими.
    Их можно удалить.
\end{proof}

\begin{example}
    Приведем в Нормальную Форму Хомского однозначную грамматику правильных скобочных последовательностей: $S \to a S b S \mid \varepsilon$

    Первым шагом добавим новый нетерминал и сделаем его стартовым:
    \begin{align*}
        S_0 & \to S                        \\
        S   & \to a S b S \mid \varepsilon
    \end{align*}
    Заменим все терминалы на новые нетерминалы:
    \begin{align*}
        S_0 & \to S                        \\
        S   & \to L S R S \mid \varepsilon \\
        L   & \to a                        \\
        R   & \to b
    \end{align*}
    Избавимся от длинных правил:
    \begin{align*}
        S_0 & \to S                     \\
        S   & \to L S' \mid \varepsilon \\
        S'  & \to S S''                 \\
        S'' & \to R S                   \\
        L   & \to a                     \\
        R   & \to b
    \end{align*}
    Избавимся от $\varepsilon$-продукций:
    \begin{align*}
        S_0 & \to S \mid \varepsilon \\
        S   & \to L S'               \\
        S'  & \to S'' \mid S S''     \\
        S'' & \to R   \mid R S       \\
        L   & \to a                  \\
        R   & \to b
    \end{align*}
    Избавимся от цепных правил:
    \begin{align*}
        S_0 & \to L S' \mid \varepsilon \\
        S   & \to L S'                  \\
        S'  & \to b \mid R S \mid S S'' \\
        S'' & \to b \mid R S            \\
        L   & \to a                     \\
        R   & \to b
    \end{align*}
\end{example}

\begin{definition}[Ослабленная нормальная форма Хомского (ОНФХ)]
    \label{defn:wCNF}
    Контекстно-свободная грамматика $\langle \Sigma, N, P, S\rangle$ находится в \emph{ослабленной Нормальной Форме Хомского}, если она содержит только правила следующего вида:
    \begin{itemize}
        \item $A \to B C$, где $A, B, C \in N$;
        \item $A \to a$, где $A \in N$, $a \in \Sigma$;
        \item $A \to \varepsilon$, где $A \in N$.
    \end{itemize}
\end{definition}

То есть ослабленная НФХ отличается от НФХ тем, что:
\begin{enumerate}
    \item $\varepsilon$ может выводиться из любого нетерминала;
    \item $S$ может появляться в правых частях правил.
\end{enumerate}


\section{Лемма о накачке}

\begin{lemma}
    Пусть $L$~--- контекстно-свободный язык над алфавитом $\Sigma$, тогда существует такое $n$, что для любого слова $\omega \in L$, $|\omega| \geq n$ найдутся слова $u,v,x,y,z\in \Sigma^*$, для которых верно: $uvxyz = \omega, vy\neq \varepsilon,|vxy|\leq n$ и для любого $k \geq 0$  $uv^kxy^kz \in L$.
\end{lemma}

\begin{proofSketch}

    \begin{enumerate}
        \item Для любого КС языка можно найти грамматику в нормальной форме Хомского.
        \item Очевидно, что если брать достаточно длинные цепочки, то в дереве вывода этих цепочек, на пути от корня к какому-то листу обязательно будет нетерминал, встречающийся минимум два раза. Если $m$~--- количество нетерминалов в НФХ, то длины $2^{m+1}$ должно хватить. Это и будет $n$ из леммы.
        \item Возьмём путь, на котором есть хотя бы дважды повторяется некоторый нетерминал. Скажем, это нетерминал  $N_1$. Пойдём от листа по этому пути. Найдём первое появление $N_1$. Цепочка, задаваемая поддеревом для этого узла~--- это $x$ из леммы.
        \item Пойдём дальше и найдём второе появление $N_1$. Цепочка, задаваемая поддеревом для этого узла~--- это $vxy$ из леммы.
        \item Теперь мы можем копировать кусок дерева между этими повторениями $N_1$ и таким образом накачивать исходную цепочку.
    \end{enumerate}
    Надо только проверить выполнение ограничений на длины.
    Нахождение разбиения и пример накачки продемонстрированы на рисунках~\ref{fig:pumping1} и~\ref{fig:pumping2}.
\end{proofSketch}

\begin{marginfigure}
    \centering
    \input{figures/cfl/pumping1.tex}
    \caption{Разбиение цепочки для леммы о накачке}
    \label{fig:pumping1}
\end{marginfigure}

\begin{marginfigure}
    \begin{center}
    \begin{subfigure}{\marginparwidth}
        \centering
        \input{figures/cfl/pumping0.tex}
        \caption{$k = 0$.}
        % \label{fig:f1}
    \end{subfigure}
    
    \begin{subfigure}{\linewidth}
        \centering
        \input{figures/cfl/pumping2.tex}
        \caption{$k = 2$.}
        % \label{fig:f2}
    \end{subfigure}
\end{center}
    \caption{Пример накачки цепочки с рисунка~\ref{fig:pumping1}}
    \label{fig:pumping2}
\end{marginfigure}

Для примера предлагается проверить неконтекстно-свободность языка $L=\{a^nb^nc^n \mid n>0\}$.

\section{Замкнутость КС языков относительно операций}

\begin{theorem}
    Контекстно-свободные языки замкнуты относительно следующих операций:
    \begin{enumerate}
        \item Объединение: если $L_1$ и $L_2$~--- контекстно-свободные языки, то и $L_3 = L_1 \cup L_2$~--- контекстно-свободный.
        \item Конкатенация: если $L_1$ и $L_2$~--- контекстно-свободные языки, то и $L_3 = L_1 \cdot L_2$~--- контекстно-свободный.
        \item Замыкание Клини: если $L_1$~--- контекстно-свободный, то и $L_2 = \bigcup\limits_{i=0}^{\infty} L_1^i $~--- контекстно-свободный.
        \item Разворот: если $L_1$~--- контекстно-свободный, то и $L_2 = {L_1}^r = \{ l^r \mid l \in L_1\}$ является контекстно-свободным.
        \item Пересечение с регулярными языками: если $L_1$~--- контекстно-свободный, а $L_2$~--- регулярный, то  $L_3 = L_1 \cap L_2$~--- контекстно-свободный.
        \item Разность с регулярными языками: если $L_1$~--- контекстно-свободный, а $L_2$~--- регулярный, то  $L_3 = L_1 \setminus L_2$~--- контекстно-свободный.
    \end{enumerate}
\end{theorem}
\begin{proof}
    Для доказательства пунктов 1--4 можно построить КС грамматику нового языка имея грамматики для исходных.
    Будем предполагать, что множества нетерминальных символов различных грамматик для исходных языков не пересекаются.
    \begin{enumerate}
        \item $G_1=\langle\Sigma_1,N_1,P_1,S_1\rangle$~--- грамматика для $L_1$, $G_1=\langle\Sigma_2,N_2,P_2,S_2\rangle$~--- грамматика для $L_2$, тогда 
        \[G_3=\langle\Sigma_1 \cup \Sigma_2, N_1 \cup N_2 \cup \{S_3\}, P_1 \cup P_2 \cup \{S_3 \to S_1 \mid S_2\} ,S_3\rangle\] ---~грамматика для $L_3$.
        \item $G_1=\langle\Sigma_1,N_1,P_1,S_1\rangle$~--- грамматика для $L_1$, $G_1=\langle\Sigma_2,N_2,P_2,S_2\rangle$~--- грамматика для $L_2$, тогда $G_3=\langle\Sigma_1 \cup \Sigma_2, N_1 \cup N_2 \cup \{S_3\}, P_1 \cup P_2 \cup \{S_3 \to S_1 S_2\} ,S_3\rangle$~--- грамматика для $L_3$.
        \item $G_1=\langle\Sigma_1,N_1,P_1,S_1\rangle$~--- грамматика для $L_1$, тогда $G_2=\langle\Sigma_1, N_1 \cup \{S_2\}, P_1 \cup \{S_2 \to S_1 S_2\ \mid \varepsilon\}, S_2\rangle$~--- грамматика для $L_2$.
        \item $G_1=\langle\Sigma_1,N_1,P_1,S_1\rangle$~--- грамматика для $L_1$, тогда $G_2=\langle\Sigma_1, N_1, \{N^i \to \omega^R \mid N^i \to \omega \in P_1 \}, S_1\rangle$~--- грамматика для $L_2$.
    \end{enumerate}

    Чтобы доказать замкнутость относительно пересечения с регулярными языками, построим по КС грамматике рекурсивный автомат $R_1$, по регулярному выражению~--- детерминированный конечный автомат $R_2$, и построим их прямое произведение $R_3$.
    Переходы по терминальным символам в новом автомате возможны тогда и только тогда, когда они возможны одновременно и в исходном рекурсивном автомате и в исходном конечном.
    За рекурсивные вызовы отвечает исходный рекурсивный автомат.
    \marginnote{TODO: Вот тут RSM очень внезапно вылез}
    Значит цепочка принимается $R_3$ тогда и только тогда, когда она принимается одновременно $R_1$ и $R_2$: так как состояния $R_3$~--- это пары из состояния $R_1$ и $R_2$, то по трассе вычислений $R_3$ мы всегда можем построить трассу для $R_1$ и $R_2$ и наоборот.

    Чтобы доказать замкнутость относительно разности с регулярным языком, достаточно вспомнить, что регулярные языки замкнуты относительно дополнения, и выразить разность через пересечение с дополнением:
    \[L_1 \setminus L_2 = L_1 \cap \overline{L_2} \qedhere\]
\end{proof}

\begin{theorem}
    Контекстно-свободные языки не замкнуты относительно следующих операций:
    \begin{enumerate}
        \item Пересечение: если $L_1$ и $L_2$~--- контекстно-свободные языки, то и $L_3 = L_1 \cap L_2$~--- не контекстно-свободный.
        \item Разность: если $L_1$ и $L_2$~--- контекстно-свободные языки, то и $L_3 = L_1 \setminus L_2$~--- не контекстно-свободный.
    \end{enumerate}
\end{theorem}

\begin{proof}
    Чтобы доказать незамкнутость относительно пресечения, рассмотрим языки $L_1 = \{a^n b^n c^k \mid n \geq 0, k \geq 0\}$ и $L_2 = \{a^k b^n c^n \mid n \geq 0, k \geq 0\}$.
    Очевидно, что $L_1$ и $L_2$~--- контекстно-свободные языки.
    Рассмотрим $L_3 = L_1 \cap L_2 = \{a^n b^n c^n \mid n \geq 0\}$.
    $L_3$ не является контекстно-свободным по лемме о накачке для контекстно-свободных языков.

    Чтобы доказать незамкнутость относительно разности проделаем следующее.
    \begin{enumerate}
        \item Рассмотрим языки $L_4 = \{a^m b^n c^k \mid m \neq n, k \geq 0\}$ и $L_5 = \{a^m b^n c^k \mid n \neq k, m \geq 0\}$.
              Эти языки являются контекстно-свободными.
              Это легко заметить, если знать, что язык $L'_4 = \{a^m b^n c^k \mid 0 \leq m < n, k \geq 0\}$ задаётся следующей грамматикой:
              \begin{align*}
                  S & \to S c & T & \to a T b \\
                  S & \to T   & T & \to T b   \\
                    &         & T & \to b.
              \end{align*}
        \item Рассмотрим язык $L_6 = \overline{L'_6} = \overline{\{a^n b^m c^k \mid n \geq 0, m \geq 0, k \geq 0\}}$.
              Данный язык является регулярным\sidenote{Предлагаем читателю самостоятельно написать регулярное выражение, задающее этот язык.}.
        \item Рассмотрим язык $L_7 = L_4 \cup L_5 \cup L_6$~--- контекстно-свободный, так как является объединением контекстно-свободных.
        \item Рассмотрим $\overline{L_7} = \{a^n b^n c^n \mid n \geq 0\} = L_3$: $L_4$ и $L_5$ задают языки с правильным порядком символов, но неравным их количеством, $L_6$ задаёт язык с неправильным порядком символов.
              Из предыдущего пункта мы знаем, что $L_3$  не является контекстно-свободным.
              \qedhere
    \end{enumerate}
\end{proof}




%\section{Вопросы и задачи}
%\begin{enumerate}
%  \item Постройте дерево вывода цепочки $w=aababb$ в грамматике $G=\langle\{a,b\},\{S\},\{S\rightarrow \varepsilon \ | \ a \ S \ b \ S \}, S \rangle$.
%  \item Постройте все левосторонние выводы цепочки $w=ababab$ в грамматике $G=\langle\{a,b\},\{S\},\{S\rightarrow \varepsilon \ | \ a \ S \ b \ | S \ S\}, S \rangle$.
%  \item Постройте все правосторонние выводы цепочки $w=ababab$ в грамматике $G=\langle\{a,b\},\{S\},\{S\rightarrow \varepsilon \ | \ a \ S \ b \ | S \ S\}, S \rangle$.
%  \item \label{t1}Постройте все деревья вывода цепочки $w=ababab$ в грамматике $G=\langle\{a,b\},\{S\},\{S\rightarrow \varepsilon \ | \ a \ S \ b \ | S \ S\}, S \rangle$, соответствующие левосторонним выводам.
%  \item \label{t2}Постройте все деревья вывода цепочки $w=ababab$ в грамматике $G=\langle\{a,b\},\{S\},\{S\rightarrow \varepsilon \ | \ a \ S \ b \ | S \ S\}, S \rangle$, соответствующие правосторонним выводам.
%\end{enumerate}

% \input{Multiple_Context-Free_Languages} % FIXME: Переписать главу
% %\input{ConjunctiveAndBooleanLanguages}
\input{FLPQ}
\setchapterpreamble[u]{\margintoc}
\chapter{Поиск путей с регулярными ограничениями}
\tikzsetfigurename{RPQ_}

\section{Достижимость между всеми парами вершин}

Через тензорное произведение.

Классическое построение пересечения автоматов строит их тензорное произведение.

Так как мы хотим отвечать ещё и на вопрос о достижимости, что нам надо ещё и транзитивное замыкание посчитать.

\section{Достижимость с несколькими источниками}

Достижимость от нескольких стартовых вершин через обход в ширину, основанный на линейной алгебре~\sidecite{9286186}.

Идея алгоритма основана на одновременном обходе в ширину графа и конечного автомата, построенного по грамматике.

В классической версии обхода в ширину, основанного на линейной алгебре, используется вектор, куда записывается фронт обхода графа.
Так, один раз перемножая этот вектор на матрицу смежности графа, можно совершать один шаг в обходе графа.
Покажем на примере, как данный метод может быть использован, когда мы накладываем дополнительные ограничения в виде регулярного языка на путь в графе.

Для этого, во-первых, предъявим булевые представления для матриц смежности графа и автомата для регулярного языка.
\marginnote{TODO: Подумать над разбиением на разделы и подразделы}
Затем, введем специальную блочно-диагональную матрицу для синхронизации обхода в ширину по двум матрицам смежности.
Далее, попробуем наивно реализовать обход в ширину, и посмотрим, почему наивная реализация может выдавать некорректный результат.
После этого перейдем к реализации обхода в ширину более продвинутым методом, который решает проблему наивного подхода.

\subsection{Пример работы алгоритма}
\begin{marginfigure}
    \caption{Входной граф.}
    \label{fig:ms_rpq_graph}
    \begin{center}
        \begin{tikzpicture}[
                shorten >=1pt,
                auto]
            \node[state] (q_0) {$0$};
            \node[state] (q_1) [below left = of q_0] {$1$};
            \node[state] (q_2) [above left = of q_0] {$2$};
            \node[state] (q_3) [right = of q_0]{$3$};

            \path[->]
            (q_0) edge [bend left] node {$b$} (q_3)
            (q_3) edge [bend left] node {$b$} (q_0)
            (q_0) edge node {$a$} (q_1)
            (q_1) edge node {$b$} (q_2)
            (q_2) edge node {$a$} (q_0);
        \end{tikzpicture}
    \end{center}
\end{marginfigure}
Возьмём граф изображенный на рисунке~\ref{fig:ms_rpq_graph}.


Его матрица смежности имеет следующий вид.
\marginnote{TODO: Точка внизу или снизу?}
\[
    G_1 =
    \begin{pmatrix}
        .     & \{a\} & .     & \{b\} \\
        .     & .     & \{b\} & .     \\
        \{a\} & .     & .     & .     \\
        \{b\} & .     & .     & .
    \end{pmatrix}
    \qquad
    G_1 =
    \begin{pmatrix}
        \cdot & \{a\} & \cdot & \{b\} \\
        \cdot & \cdot & \{b\} & \cdot \\
        \{a\} & \cdot & \cdot & \cdot \\
        \{b\} & \cdot & \cdot & \cdot
    \end{pmatrix}
\]

Её булева декомпозиция по каждому символу выглядит следующим образом.
\[
    G_{0\_a} = \begin{pmatrix}
        0 & 1 & 0 & 0 \\
        0 & 0 & 0 & 0 \\
        1 & 0 & 0 & 0 \\
        0 & 0 & 0 & 0 \\
    \end{pmatrix} \qquad
    G_{0\_b} = \begin{pmatrix}
        0 & 0 & 0 & 1 \\
        0 & 0 & 1 & 0 \\
        0 & 0 & 0 & 0 \\
        1 & 0 & 0 & 0 \\
    \end{pmatrix}
\]

Зададим ограничения с помощью регулярного выражения $b^*ab$, которое представляется автоматом из трех последовательных состояний.
\begin{center}
    \begin{tikzpicture}[shorten >=1pt,on grid,auto]
        \node[state, initial]   (q_0) at (0,0)  {$0$};
        \node[state]             (q_1) at (2,0)  {$1$};
        \node[state, accepting] (q_2) at (4,0)  {$2$};
        \path[->]
        (q_0) edge  node {$a$} (q_1)
        (q_1) edge  node {$b$} (q_2);
        \draw (q_0) edge[loop above]  node {$b$} (q_0);
    \end{tikzpicture}
\end{center}

Автомат может быть задан матрицей смежности (с дополнительной информацией о стартовых и финальных состояниях).

Для регулярного выражения $b^*ab$ матрица смежности выглядит следующим образом (при этом нужно запомнить, что состояние $0$ является начальным, $2$~--- конечным).
\[
    G_2 =
    \begin{pmatrix}
        \{b\} & \{a\} & .     \\
        .     & .     & \{b\} \\
        .     & .     & .
    \end{pmatrix}
\]

Нам будет необходима булева декомпозиция этой матрицы, и она выглядит следующим образом.
\[
    R_{0\_a} = \begin{pmatrix}
        0 & 1 & 0 \\
        0 & 0 & 0 \\
        0 & 0 & 0
    \end{pmatrix} \qquad
    R_{0\_b} = \begin{pmatrix}
        1 & 0 & 0 \\
        0 & 0 & 1 \\
        0 & 0 & 0
    \end{pmatrix}
\]

\NewDocumentCommand{\MDa}{}{\begin{pNiceArray}[margin]{ccc|cccc}
        0 & 1 & 0 & 0 & 0 & 0 & 0 \\
        0 & 0 & 0 & 0 & 0 & 0 & 0 \\
        0 & 0 & 0 & 0 & 0 & 0 & 0 \\
        \midrule
        0 & 0 & 0 & 0 & 1 & 0 & 0 \\
        0 & 0 & 0 & 0 & 0 & 0 & 0 \\
        0 & 0 & 0 & 1 & 0 & 0 & 0 \\
        0 & 0 & 0 & 0 & 0 & 0 & 0
    \end{pNiceArray}}
\NewDocumentCommand{\MDb}{}{\begin{pNiceArray}[margin]{ccc|cccc}
        1 & 0 & 0 & 0 & 0 & 0 & 0 \\
        0 & 0 & 1 & 0 & 0 & 0 & 0 \\
        0 & 0 & 0 & 0 & 0 & 0 & 0 \\
        \midrule
        0 & 0 & 0 & 0 & 0 & 0 & 1 \\
        0 & 0 & 0 & 0 & 0 & 1 & 0 \\
        0 & 0 & 0 & 0 & 0 & 0 & 0 \\
        0 & 0 & 0 & 1 & 0 & 0 & 0
    \end{pNiceArray}}

Для синхронизации обхода составим набор блочно-диагональных матриц, каждая из которых~--- это прямая сумма двух матриц: $D_{0\_a} = R_{0\_a} \oplus G_{0\_a}$ и $D_{0\_a} = R_{0\_b} \oplus G_{0\_b}$.
\[
    D_{0\_a} = \MDa \quad
    D_{0\_b} = \MDb
\]

Пусть мы решаем частный случай задачи достижимости с несколькими стартовыми вершинами (multiple-source)~--- достижимость с одной стартовой вершиной (single-source).

Пусть единственной начальной вершиной в графе будет вершина $0$.

Теперь создадим вектор
$v = \begin{pNiceArray}[]{ccc|cccc}
        1 & 0 & 0 & 1 & 0 & 0 & 0
    \end{pNiceArray}$,
где в первой части стоит единица на месте начального состояния $0$ в автомате.
Во второй части содержится фронт обхода графа, на первом шаге это всегда множество стартовых вершин.
В данном случае единица стоит на месте единственной стартовой вершины~--- $0$.

Совершим один шаг в обходе графа и получим новый фронт обхода графа.
\begin{widepar}
    \begin{gather*}
        a: \begin{pNiceArray}[]{ccc|cccc}
            1 & 0 & 0 & 1 & 0 & 0 & 0
        \end{pNiceArray} \cdot \MDa =
        \begin{pNiceArray}[]{ccc|cccc}
            0 & 1 & 0 & 0 & 1 & 0 & 0
        \end{pNiceArray}
        \\
        b: \begin{pNiceArray}[]{ccc|cccc}
            1 & 0 & 0 & 1 & 0 & 0 & 0
        \end{pNiceArray} \cdot \MDb  =
        \begin{pNiceArray}[]{ccc|cccc}
            1 & 0 & 0 & 0 & 0 & 0 & 1
        \end{pNiceArray}
    \end{gather*}
\end{widepar}

Сложим два полученных вектора, чтобы получить новый фронт обхода графа:
\begin{widepar}
    \[
        \begin{pNiceArray}[]{ccc|cccc}
            0 & 1 & 0 & 0 & 1 & 0 & 0
        \end{pNiceArray} +
        \begin{pNiceArray}[]{ccc|cccc}
            1 & 0 & 0 & 0 & 0 & 0 & 1
        \end{pNiceArray} =
        \begin{pNiceArray}[]{ccc|cccc}
            1 & 1 & 0 & 0 & 1 & 0 & 1
        \end{pNiceArray}.
    \]
\end{widepar}

То есть в наш фронт
$\begin{pmatrix}
        0 & 1 & 0 & 1
    \end{pmatrix}$
попали вершины 1 и 3 соотвественно.
А именно, мы совершили следующие переходы в графе и автомате.

\begin{minipage}{0.45\textwidth}
    \begin{center}
        \begin{tikzpicture}[node distance=2cm,shorten >=1pt,on grid,auto]
            \node[state, red] (q_0)   {$1$};
            \node[state] (q_1) [above=of q_0] {$2$};
            \node[state] (q_2) [right=of $(q_0)!0.5!(q_1)$] {$0$};
            \node[state, red] (q_3) [right=of q_2] {$3$};
            \path[->, red]
            (q_2) edge  node[pos=0.7] {a} (q_0)
            (q_2) edge[bend left]  node[above] {b} (q_3);
            \path[->]
            (q_0) edge  node {b} (q_1)
            (q_1) edge  node[pos=0.3] {a} (q_2)
            (q_3) edge[bend left]  node {b} (q_2);
        \end{tikzpicture}
    \end{center}
\end{minipage}
\begin{minipage}{0.45\textwidth}
    \begin{center}
        \begin{tikzpicture}[shorten >=1pt,on grid,auto]
            \node[state, draw=none]      (q_3) at (0,0)  {$ $}; % empty node for alignment
            \node[state, initial, red]   (q_0) at (0,1)  {$0$};
            \node[state, red]            (q_1) at (2,1)  {$1$};
            \node[state, accepting]      (q_2) at (4,1)  {$2$};
            \path[->, red]
            (q_0) edge  node {$a$} (q_1);
            \path[->]
            (q_1) edge  node {$b$} (q_2);
            \draw (q_0) edge[loop above, red]  node {$b$} (q_0);
        \end{tikzpicture}
    \end{center}
\end{minipage}

Совершим еще один шаг алгоритма.
Теперь вектор $v$, на который мы умножаем матрицы, имеет следующий вид
$\begin{pNiceArray}[]{ccc|cccc}
        1 & 1 & 0 & 0 & 1 & 0 & 1
    \end{pNiceArray}$.
\begin{widepar}
    \begin{gather*}
        a: \begin{pNiceArray}[]{ccc|cccc}
            1 & 1 & 0 & 0 & 1 & 0 & 1
        \end{pNiceArray} \cdot \MDa =
        \begin{pNiceArray}[]{ccc|cccc}
            0 & 1 & 0 & 0 & 0 & 0 & 0
        \end{pNiceArray} \displaybreak[0] \\ % FIXME: remove \displaybreak[0]
        b: \begin{pNiceArray}[]{ccc|cccc}
            1 & 1 & 0 & 0 & 1 & 0 & 1
        \end{pNiceArray} \cdot \MDb =
        \begin{pNiceArray}[]{ccc|cccc}
            1 & 0 & 1 & 1 & 0 & 1 & 0
        \end{pNiceArray} \\
        \begin{pNiceArray}[]{ccc|cccc}
            0 & 1 & 0 & 0 & 0 & 0 & 0
        \end{pNiceArray} +
        \begin{pNiceArray}[]{ccc|cccc}
            1 & 0 & 1 & 1 & 0 & 1 & 0
        \end{pNiceArray} =
        \begin{pNiceArray}[]{ccc|cccc}
            1 & 1 & 1 & 1 & 0 & 1 & 0
        \end{pNiceArray}
    \end{gather*}
\end{widepar}
То есть в наш фронт
$\begin{pmatrix}
        1 & 0 & 1 & 0
    \end{pmatrix}$
попали вершины 0 и 2 соотвественно.
Мы совершили следующие переходы в графе и автомате.

\begin{minipage}{0.45\textwidth}
    \begin{center}
        \begin{tikzpicture}[node distance=2cm,shorten >=1pt,on grid,auto]
            \node[state, gray] (q_0)   {$1$};
            \node[state, teal] (q_1) [above=of q_0] {$2$};
            \node[state, teal] (q_2) [right=of $(q_0)!0.5!(q_1)$] {$0$};
            \node[state, gray] (q_3) [right=of q_2] {$3$};
            \path[->, gray]
            (q_2) edge  node[pos=0.7] {a} (q_0)
            (q_2) edge[bend left]  node[above] {b} (q_3);
            \path[->, red]
            (q_0) edge  node {b} (q_1)
            (q_3) edge[bend left]  node {b} (q_2);
            \path[->]
            (q_1) edge  node[pos=0.3] {a} (q_2);
        \end{tikzpicture}
    \end{center}
\end{minipage}
\begin{minipage}{0.45\textwidth}
    \begin{center}
        \begin{tikzpicture}[shorten >=1pt,on grid,auto]
            \node[state, draw=none]            (q_3) at (0,0)  {$ $}; % empty node for alignment
            \node[state, initial, red]         (q_0) at (0,1)  {$0$};
            \node[state, gray]                 (q_1) at (2,1)  {$1$};
            \node[state, accepting, teal]      (q_2) at (4,1)  {$2$};
            \path[->, gray]
            (q_0) edge  node {$a$} (q_1);
            \path[->, red]
            (q_1) edge  node {$b$} (q_2);
            \draw (q_0) edge[loop above, red]  node {$b$} (q_0);
        \end{tikzpicture}
    \end{center}
\end{minipage}

При этом, можно заметить, что мы достигли конечной вершины в автомате.
Последний элемент левой части результирующего вектора
$\begin{pNiceArray}[]{ccc|cccc}
        1 & 1 & \textcolor{red}{1} & 1 & 0 & 1 & 0
    \end{pNiceArray}$
отвечает за состояние 2, которое является конечным.
А значит, обход необходимо остановить, и текущие вершины фронта обхода графа записать в ответ.

Таким образом, вершины графа 0 и 2 являются ответом.
Однако вершина 0~--- лишняя.
Регулярное выражение $b^*ab$ не подразумевает, что вершина 0 в графе может быть достигнута.
Она могла бы быть достигнута по пустой строке в случае, если бы регулярное выражение имело вид $b^*$ или по строке $aba$ в случае, если бы регулярное выражение имело вид $b^*aba$.

Это произошло, потому что в векторе $v$ должна кодироваться информация о паре~--- вершине графа и состоянии автомата.
Достигнув вершины 0, мы оказались в конечном состоянии автомата, которое было получено с помощью другой вершины~--- вершины 2.

Эту проблему можно решить, закодировав информацию о каждой такой паре в несколько векторов $v$, и ограничив левую часть вектора $v$ таким образом, чтобы в ней всегда была лишь одна единица.

Тогда мы получим, что вектор $v$ вида
$\begin{pNiceArray}[]{ccc|cccc}
        1 & 0 & 0 & 1 & 0 & 1 & 0
    \end{pNiceArray}$
будет хранить информацию о парах $(0, 0)$ и $(0, 2)$, где первый элемент пары~--- состояние автомата, а второй~--- вершина графа.

Аналогично, вектор $v$ вида
$\begin{pNiceArray}[]{ccc|cccc}
        0 & 1 & 0 & 0 & 1 & 1 & 0
    \end{pNiceArray}$
кодирует информацию о парах $(1, 1)$ и $(1, 2)$.
Вектор $v$ вида
$\begin{pNiceArray}[]{ccc|cccc}
        0 & 0 & 1 & 0 & 0 & 1 & 1
    \end{pNiceArray}$
кодирует информацию о парах $(2, 2)$ и $(2, 3)$.

Таким образом, мы будем понимать, в каком состоянии автомата мы находимся для каждой из вершин фронта обхода графа.

Рассмотрим, как это применяется в разработанном алгоритме, который представлен далее.

Предлагается \enquote{расклеить} $v$ в матрицу $M$, состоящую из трех векторов, добавив два вектора $\begin{pNiceArray}[]{ccc|cccc}
        0 & 1 & 0 & 0 & 0 & 0 & 0
    \end{pNiceArray}$
и $\begin{pNiceArray}[]{ccc|cccc}
        0 & 0 & 1 & 0 & 0 & 0 & 0
    \end{pNiceArray}$.
Во второй части этих векторов стоят нули, так как
$\begin{pmatrix}
        0 & 1 & 0
    \end{pmatrix}$
и
$\begin{pmatrix}
        0 & 0 & 1
    \end{pmatrix}$
кодируют состояния автомата 1 и 2, которые не являются начальными.
\NewDocumentCommand{\MMf}{}{\begin{pNiceArray}[]{ccc|cccc}
        1 & 0 & 0 & 1 & 0 & 0 & 0 \\
        0 & 1 & 0 & 0 & 0 & 0 & 0 \\
        0 & 0 & 1 & 0 & 0 & 0 & 0
    \end{pNiceArray}}
\[M = \MMf\]

И совершать обход тем же самым образом, но сохранив с помощью матрицы $M$ дополнительную информацию о парах (состояние, вершина).
\begin{widepar}
    \begin{gather*}
        a: \MMf \cdot \MDa = \begin{pNiceArray}[]{ccc|cccc}
            0 & 1 & 0 & 0 & 1 & 0 & 0 \\
            0 & 0 & 0 & 0 & 0 & 0 & 0 \\
            0 & 0 & 0 & 0 & 0 & 0 & 0
        \end{pNiceArray} \\
        b: \MMf \cdot \MDb = \begin{pNiceArray}[]{ccc|cccc}
            1 & 0 & 0 & 0 & 0 & 0 & 1 \\
            0 & 0 & 1 & 0 & 0 & 0 & 0 \\
            0 & 0 & 0 & 0 & 0 & 0 & 0
        \end{pNiceArray}
    \end{gather*}
\end{widepar}

Для того, чтобы левая часть матрицы $M$ всегда оставалось единичной, нужно трансформировать в ней строчки особым образом.
Для этого нужно складывать только те вектора в правой части матрицы $M$, у которых в левой части единицы стоят на одинаковых позициях.
После чего переставлять строчки в $M$ так, чтобы левая часть матрицы $M$ принимала единичный вид.
Вектора с пустой левой частью нас при этом не интересуют.

Тогда правая часть матрицы $M$ будет кодировать текущий фронт обхода графа.

В нашем примере матрица $M$ для следующего шага обхода выглядит следующим образом.
\NewDocumentCommand{\MMs}{}{\begin{pNiceArray}[]{ccc|cccc}
        1 & 0 & 0 & 0 & 0 & 0 & 1 \\
        0 & 1 & 0 & 0 & 1 & 0 & 0 \\
        0 & 0 & 1 & 0 & 0 & 0 & 0
    \end{pNiceArray}}
\[M = \MMs\]

Видно, что во фронт обхода графа попали вершины 1 и 3.
В вершину 1 мы попали в состоянии 1, в вершину 3~--- в состоянии 0.

Совершаются следующие переходы в графе и автомате.

\begin{minipage}{0.45\textwidth}
    \begin{center}
        \begin{tikzpicture}[node distance=2cm,shorten >=1pt,on grid,auto]
            \node[state, red] (q_0)   {$1$};
            \node[state] (q_1) [above=of q_0] {$2$};
            \node[state] (q_2) [right=of $(q_0)!0.5!(q_1)$] {$0$};
            \node[state, red] (q_3) [right=of q_2] {$3$};
            \path[->, red]
            (q_2) edge  node[pos=0.7] {a} (q_0)
            (q_2) edge[bend left]  node[above] {b} (q_3);
            \path[->]
            (q_0) edge  node {b} (q_1)
            (q_1) edge  node[pos=0.3] {a} (q_2)
            (q_3) edge[bend left]  node {b} (q_2);
        \end{tikzpicture}
    \end{center}
\end{minipage}
\begin{minipage}{0.45\textwidth}
    \begin{center}
        \begin{tikzpicture}[shorten >=1pt,on grid,auto]
            \node[state, draw=none]      (q_3) at (0,0)  {$ $}; % empty node for alignment
            \node[state, initial, red]   (q_0) at (0,1)  {$0$};
            \node[state, red]            (q_1) at (2,1)  {$1$};
            \node[state, accepting]      (q_2) at (4,1)  {$2$};
            \path[->, red]
            (q_0) edge  node {$a$} (q_1);
            \path[->]
            (q_1) edge  node {$b$} (q_2);
            \draw (q_0) edge[loop above, red]  node {$b$} (q_0);
        \end{tikzpicture}
    \end{center}
\end{minipage}

Сделаем еще один шаг алгоритма и придем к конечному состоянию в автомате.
\NewDocumentCommand{\MMt}{}{\begin{pNiceArray}[]{ccc|cccc}
        1 & 0 & 0 & 1 & 0 & 0 & 0 \\
        0 & 1 & 0 & 0 & 0 & 0 & 0 \\
        0 & 0 & 1 & 0 & 0 & 1 & 0
    \end{pNiceArray}}

\begin{widepar}
    \begin{gather*}
        \MMs \cdot \MDa = \begin{pNiceArray}[]{ccc|cccc}
            0 & 1 & 0 & 0 & 0 & 0 & 0 \\
            0 & 0 & 0 & 0 & 0 & 0 & 0 \\
            0 & 0 & 0 & 0 & 0 & 0 & 0
        \end{pNiceArray} \\
        \MMs \cdot \MDb = \begin{pNiceArray}[]{ccc|cccc}
            1 & 0 & 0 & 1 & 0 & 0 & 0 \\
            0 & 0 & 1 & 0 & 0 & 1 & 0 \\
            0 & 0 & 0 & 0 & 0 & 0 & 0
        \end{pNiceArray} \\
        M = \MMt
    \end{gather*}
\end{widepar}

Видно, что мы достигли вершины 2 графа в конечном состоянии 2 автомата.
При этом вершина 0 графа так же достигнута, как и в наивном варианте алгоритма, но теперь известно, что это происходит в состоянии 0 автомата.

\begin{minipage}{0.45\textwidth}
    \begin{center}
        \begin{tikzpicture}[node distance=2cm,shorten >=1pt,on grid,auto]
            \node[state, gray] (q_0)   {$1$};
            \node[state, teal] (q_1) [above=of q_0] {$2$};
            \node[state, red] (q_2) [right=of $(q_0)!0.5!(q_1)$] {$0$};
            \node[state, gray] (q_3) [right=of q_2] {$3$};
            \path[->, gray]
            (q_2) edge  node[pos=0.7] {a} (q_0)
            (q_2) edge[bend left]  node[above] {b} (q_3);
            \path[->, red]
            (q_0) edge  node {b} (q_1)
            (q_3) edge[bend left]  node {b} (q_2);
            \path[->]
            (q_1) edge  node[pos=0.3] {a} (q_2);
        \end{tikzpicture}
    \end{center}
\end{minipage}
\begin{minipage}{0.45\textwidth}
    \begin{center}
        \begin{tikzpicture}[shorten >=1pt,on grid,auto]
            \node[state, draw=none]            (q_3) at (0,0)  {$ $}; % empty node for alignment
            \node[state, initial, red]         (q_0) at (0,1)  {$0$};
            \node[state, gray]                 (q_1) at (2,1)  {$1$};
            \node[state, accepting, teal]      (q_2) at (4,1)  {$2$};
            \path[->, gray]
            (q_0) edge  node {$a$} (q_1);
            \path[->, red]
            (q_1) edge  node {$b$} (q_2);
            \draw (q_0) edge[loop above, red]  node {$b$} (q_0);
        \end{tikzpicture}
    \end{center}
\end{minipage}

Таким образом, в ответ попадает вершина 2.

Перейдем к формальному описанию алгоритма.

\subsection{Формальное описание алгоритма}

Алгоритм принимает на вход граф $\mscrG$, детерминированный конечный автомат $\mscrR$, описывающий регулярную грамматику, и множество начальных вершин $V_{\mathrm{src}}$ графа.

Граф $\mscrG$ и автомат $\mscrR$ можно представить в виде булевых матриц смежности.
Так, в виде словаря для каждой метки графа заводится булева матрица смежности, на месте $(i, j)$ ячейки которой стоит 1, если $i$ и $j$ вершины графа соединены ребром данной метки.
Такая же операция проводится для автомата грамматики $\mscrR$.

Далее, мы оперируем с двумя словарями, где ключом является символ метки ребра графа или символ алфавита автомата, а значением~--- соответствующая им булевая матрица.

Для каждого символа из пересечения этих множеств строится матрица $\mfrakD$, как прямая сумма булевых матриц.
То есть, строится матрица $\mfrakD = \mathrm{Bool}_{\mscrR_a} \oplus \mathrm{Bool}_{\mscrG_a}$, которая определяется как
\[
    \mfrakD = \begin{pmatrix}
        \mathrm{Bool}_{\mscrR_a} & 0                        \\
        0                        & \mathrm{Bool}_{\mscrG_a}
    \end{pmatrix},
\]
где $\mscrR_{a}$ и $\mscrG_{a}$ матрицы смежности соответствующих символов автомата грамматики $\mscrR$ и графа $\mscrG$ для символа $a \in A_\mscrR \cap A_\mscrG$, $A_\mscrR \cap A_\mscrG$~--- пересечение алфавитов.
Такая конструкция позволяет синхронизировать алгоритм обхода в ширину одновременно для графа и грамматики.

Далее вводится матрица $M$, хранящая информацию о фронте обхода графа.
\marginnote{TODO: Надо подумать над обозначениями. Почему $Id$, а не $E$? Может быть вообще надо эти обозначения вводить сразу в примере.}
Она нужна для выделения множества пройденных вершин и не допускает зацикливание алгоритма.
\[
    M^{k \times (k + n)} = \begin{pNiceArray}[]{c|c}
        Id_k & Matrix_{k \times n }
    \end{pNiceArray},
\]
где $Id_k$~--- единичная матрица размера $k$, $k$~--- количество вершин в автомате $\mscrR$, $Matrix_{k \times n }$~--- матрица, хранящая в себе маску пройденных вершин в автомате графа, $n$~--- количество вершин в графе $\mscrG$.

\subsection{Выходные данные}

На выходе строится множество $\mscrP$ пар вершин $(v, w)$ графа $\mscrG$ таких, что вершина $w$ достижима из множества начальных вершин, при этом $v \in V_{\mathrm{src}}$, $w \not\in V_{\mathrm{src}}$.
Это множество представляется в виде матрицы размера $|V| \times |V|$, где $(i,j)$ ячейка содержит 1, если пара вершин с индексами $(i, j) \in \mscrP$.

\subsection{Процесс обхода графа}

Алгоритм обхода заключается в последовательном умножении матрицы $M$ текущего фронта на матрицу $\mfrakD$.
В результате чего, находится матрица $M'$ содержащая информацию о вершинах, достижимых на следующем шаге.
Далее, с помощью операций перестановки и сложения векторов $M'$ преобразуется к виду матрицы $M$ и присваивается ей.
Итерации продолжаются пока $M'$ содержит новые вершины, не содержащиеся в $M$. На листинге~\ref{BFSRPQ1} представлен этот алгоритм.

%% FIXME: Переверстать алгоритм
% \begin{algorithm}[t]
%     \caption{Алгоритм достижимости в графе с регулярными ограничениями на основе поиска в ширину, выраженный с помощью операций матричного умножения}\label{BFSRPQ1}
%     \begin{algorithmic}[1]
%         \Procedure{BFSBasedRPQ}{$\mscrR=\langle Q, \Sigma, P, F, q \rangle,\mscrG=\langle V, E, L \rangle, V_{\mathrm{src}}$}
%         \State $\mscrP\gets~${Матрица смежности графа}
%         \State $\mfrakD\gets \mathrm{Bool}_\mscrR \bigoplus \mathrm{Bool}_\mscrG$\Comment{Построение матриц $\mfrakD$}
%         \State $M\gets CreateMasks(|Q|,|V|)$ \Comment{Построение матрицы $M$}
%         \State $M'\gets SetStartVerts(M, V_{\mathrm{src}})$  \Comment{Заполнение нач. вершин}

%         \While{Матрица~$M$~меняется}{}
%         \State $M\gets M'\langle\neg M\rangle$\Comment{Применение комплементарной маски}
%         \ForAll{$a\in (\Sigma \cap L)$}
%         \State $M'\gets M~$any.pair$~\mfrakD$
%         \Comment{Матр. умножение в полукольце}
%         \State $M'\gets TransformRows(M')$\label{TransformRows}
%         \Comment{Приведение $M'$ к виду $M$}
%         \EndFor
%         \State {$Matrix\gets extractRightSubMatrix(M')$}
%         \State $V\gets Matrix.reduceVector()$ \Comment{Сложение по столбцам}
%         \For{$k \in 0\dots|V_{\mathrm{src}}|-1$}
%         \State $W\gets\mscrP.getRow(k)$
%         \State $\mscrP.setRow(k, V+W)$
%         \EndFor
%         \EndWhile
%         \State \textbf{return} $\mscrP$
%         \EndProcedure
%     \end{algorithmic}
% \end{algorithm}

В алгоритме~\ref{BFSRPQ1}, в~\ref{TransformRows} строке происходит трансформация строчек в матрице $M'$.
Это делается для того, чтобы представить полученную во время обхода матрицу $M'$, содержащую новый фронт, в виде матрицы $M$.
Для этого требуется так переставить строчки $M'$, чтобы она содержала корректные по своему определению значения.
То есть, имела единицы на главной диагонали, а все остальные значения в первых $k$ столбцах были нулями.
Подробнее эта процедура описана в листинге~\ref{AlgoTransformRows}.

%% FIXME: Переверстать алгоритм
% \begin{algorithm}[H]
%     \caption{Алгоритм трансформации строчек}\label{AlgoTransformRows}
%     \begin{algorithmic}[1]
%         \Procedure{TransformRows}{$M$}
%         \State{$T \gets extractLeftSubMatrix(M)$}
%         \State{$Ix, Iy \gets$ итераторы по индексам ненулевых элементов $T$}
%         \For{$i \in 0\dots|Iy|$}
%         \State{$R\gets M.getRow(Ix[i])$}
%         \State{$M'.setRow(Iy[i], R + M'.getRow(Iy[i]))$}
%         \EndFor
%         \EndProcedure
%     \end{algorithmic}
% \end{algorithm}

\subsection{Модификации алгоритма}

Рассмотрим $V_{\mathrm{src}}$~--- множество начальных вершин, состоящее из $r$ элементов.
Для каждой начальной вершины $V_{\mathrm{src}}^i \in V_{\mathrm{src}}$ отметим соответствующие индексы в матрице $M$ единицами, получив матрицу $M(V_{\mathrm{src}}^i)$, и построим матрицу $\mfrakM$ следующим образом.
\[
    \mfrakM^{(k*r) \times (k + n)} = \begin{pmatrix}
        M(V_{\mathrm{src}}^1) \\
        M(V_{\mathrm{src}}^2) \\
        M(\dots)              \\
        M(V_{\mathrm{src}}^r)
    \end{pmatrix}
\]

Матрица $\mfrakM$ собирается из множества матриц $M(V_{\mathrm{src}}^i)$ и позволяет хранить информацию о том, из какой начальной вершины достигаются новые вершины во время обхода.

%% FIXME: Переверстать алгоритм
% \begin{algorithm}[t]
%     \caption{Модификация алгоритма для поиска конкретной исходной вершины}\label{BFSRPQ2}
%     \begin{algorithmic}[1]
%         \Procedure{BFSBasedRPQ}{$\mscrR=\langle Q, \Sigma, P, F, q \rangle,\mscrG=\langle V, E, L \rangle, V_{\mathrm{src}}$}
%         \State $\mscrP\gets~${Матрица смежности графа}
%         \State $\mfrakD\gets \mathrm{Bool}_\mscrR \bigoplus \mathrm{Bool}_\mscrG$
%         \State $\mfrakM\gets CreateMasks(|Q|,|V|)$
%         \State $\mfrakM'\gets SetStartVerts(\mfrakM, V_{\mathrm{src}})$

%         \While{Матрица~$\mfrakM$~меняется}{}
%         \State $\mfrakM\gets \mfrakM'\langle\neg\mfrakM\rangle$
%         \ForAll{$a\in (\Sigma \cap L)$}
%         \State $\mfrakM'\gets \mfrakM~$any.pair$~\mfrakD$
%         \ForAll{$M \in \mfrakM'$}
%         \State $M\gets TransformRows(M)$
%         \EndFor
%         \EndFor
%         \ForAll{$M_k \in \mfrakM'$}
%         \State $Matrix\gets extractSubMatrix(M)$
%         \State $V\gets Matrix.reduceVector()$
%         \State $W\gets\mscrP.getRow(k)$
%         \State $\mscrP.setRow(k, V+W)$
%         \EndFor
%         \EndWhile
%         \State \textbf{return} $\mscrP$
%         \EndProcedure
%     \end{algorithmic}
% \end{algorithm}

В листинге~\ref{BFSRPQ2} представлен модифицированный алгоритм.
Основное его отличие заключается в том, что для каждой достижимой вершины находится конкретная исходная вершина, из которой начинался обход.

Таким образом, алгоритмы~\ref{BFSRPQ1} и~\ref{BFSRPQ2} решают сформулированные в пункте~\ref{sec:3.3} задачи достижимости.

% %\input{CFPQ}
\input{CYK_for_CFPQ}
% \input{Matrix-based_CFPQ}
% \input{TensorProduct}
% \input{SPPF}
% \input{GLL-based_CFPQ}
% \input{GLR-based_CFPQ}
% %\input{CombinatorsForCFPQ}
% \input{Multiple_Context-Free_Language_Reachability} % FIXME: Исправить главу
% %\input{DerivativesForCFPQ}
% %\input{CFPQ_to_Datalog}
% \input{Conclusion}

\backmatter
\setchapterstyle{plain}

\printbibliography

\end{document}
